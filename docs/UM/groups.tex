At the most fundamental level a group in ADER is a list of proportions. This
list describes the relative proportions of elements within the group. These
elements may or may not have specified isotopic proportions and elements
without a specified isotopic composition are permitted in the same group as
elements \textit{with} a specified isotopic composition. Consider a group which
specifies that fluorine be four times as abundant as uranium, within the group.
This group could be used to describe a single molecule of \ce{UF4} or, just as
equivalently, one mol of uranium and four mols of fluorine all in a bucket
on a lab bench. A group is not defined by a quantity of material, but rather
a recipe for a material. Just like defining materials in SERPENT2, the
proportions of an element within a group are normalized to unity and the
proportions of an isotope within an element are normalized to unity.
As such groups in ADER are defined according to four
methods A, B, C, or D as seen bellow.\\

\noindent Method A: 
\begin{lt}
grp [Name] 
[Element_1] [Element_1_frac] 
... 
<Element_n> (Element_n_frac)
\end{lt}

\noindent Method B:
\begin{lt}
grp [Name] 
[Element_1] [Element_1_frac] <isos> (n_isos)
(Iso_1) (Iso_1_frac)
<Iso_n> (Iso_n_frac)
...
<Element_x> [Element_x_frac] <isos> (b_isos)
(Iso_1) (Iso_1_frac)
<Iso_b> (Iso_b_frac)
\end{lt}

\noindent Method C:
\begin{lt}
grp [Name] 
[Element_1] [Element_1_frac] 
...
<Element_x> [Element_x_frac] <isos> (b_isos)
(Iso_1) (Iso_1_frac)
<Iso_b> (Iso_b_frac)
\end{lt}

\noindent Method D:
\begin{lt}
grp [Name] [sum]
[Name_of_group_1] 1
...
<Name_of_group_n> 1
\end{lt}

The keyword \texttt{grp} begins the group section input. All input following 
this word, until the next keyword is found, is taken as input for the
aforementioned \texttt{grp}.
The \texttt{Name} entry is an identifier by which the group will be located
throughout the code - as such all groups require a unique name.
In method A a group is defined only by the proportions of its elements. 
The relative abundance of each element within the group is denoted by
\texttt{Element\_x\_frac} while the values for \texttt{Element\_x} must be
the alphabetic periodic table shorthand for that element, Ag for gold as an
example. Elements with no prescribed isotopic composition are permitted to take
on any combination of their own isotopes.

In method B a group is defined in which each element has a list of specified 
isotopic proportions. The values for \texttt{Iso\_n} must be the alphanumeric
name of the isotope in the form ``Alpha-A", or U-233 for \ce{^{233}U}
The proportions entered, \texttt{Iso\_n\_frac},
are of the isotope relative to the 
element as a whole. In method C a group is defined in which some of its elements
have a specified list of isotopic proportions and some do not. Those elements
which do not have a prescribed list of isotopic proportions are permitted to
have any combination of their own isotopes. Lastly, in method D a new type
of group is introduced - the summation group. A summation group is defined as
the sum of the groups named in its list of component groups. These component
groups may be defined using methods A, B, C, or D, allowing for nested summation
groups. The value which follows a group name in a summation group definition 
may be any value greater than 0; however, any value other than 1 could
lead to a non-physical answer in the simulation. Please see section
\ref{sec:theory} for a more detailed explanation.

The \texttt{Element\_x\_frac} values for each group are all summed together
and then normalized to this sum. This same procedure is done for all
isotopic values within an element's list. For example, consider the group
\texttt{gUF4} defined below, it is composed of 20\% uranium and 80\% fluorine.
The fluorine is permitted to have any combination of fluorine isotopes while
the uranium in this group is specified to be 5\% \ce{^233U} and 95\% \ce{^238U}.This is an example of method C of group definition.

\begin{li}
grp gUF4
U   1   isos    2
U-233   5
U-238   95
F   4
\end{li}

The group \texttt{gFLiBe} is defined below as an example of method A, group
\texttt{gUF3} is defined below as an example of method B, while group
\texttt{gUF} is defined below as an example of method D. These groups will be
used throughout this manual in examples. 

\begin{li}
grp gFLiBe
Li  71.7
Be  16
F   103.7

grp gUF3
F   3   isos    1
F-19    1
U   1   isos    1
U-233   1

grp gUF sum
gUF4    1
gUF3    1
\end{li}

\textit{The following paragraphs assume some level of familiarity with ADER and
will likely be unclear until sections \ref{sec:conditions_blocks} and 
\ref{sec:streams} have been reviewed.}

Groups are used to define four structures in ADER; ranges, ratios, streams, and
summation groups. These structures are covered in later sections of this manual
but their interaction with groups requires a comment here. When a group is
used in a range or ratio structure, or is used in a summation group which is
part of a range or ratio structure, and this structure is attached to a material
by a conditions block, the group in question is added to the material's list
of possible recipes. These recipes are the options into which the material's
constituent isotopes may be sorted during the optimization step, discussed
in section \ref{sec:theory_opt}. Range and ratio structures, in the same
material, make use of the same group list. That is to say, in the following
example seen below, in which the conditions block \texttt{example\_block} is
attached to the material \texttt{FuelSalt}, 
the group \texttt{gFLiBe} must \textbf{both} be 
between 20 and 80\% of \texttt{FuelSalt}'s atomic density \textbf{and} be 
four times
as abundant as the group \texttt{gUF4}. There are \textbf{not} now two copies of
the group \texttt{gFLiBe} assigned to the material \texttt{FuelSalt}.
The \texttt{rng} and \texttt{rto} structures each refer to the \textbf{same} 
group  within a single material, 
in this instance the material \texttt{FuelSalt}.

\begin{li}
conditions example_block
rng gFLiBe min 0.2 max 0.8
rto gFLiBe val 4 grp2 gUF4

mat FuelSalt -2.805 vol 1 burn 0 ader cnd example_block
3006.06c    0.00028
...
\end{li}

The groups used to define group-class streams are used solely as recipes. 
The groups used to define group-class streams \textit{only} specify the
relative proportions of the mass carried by the stream in question. For a
stream to be connected to a material, that material \textit{does not} need to
have that group in its list of possible recipes. For instance, 
\texttt{FuelSalt}
to which the conditions block above, \texttt{example\_block}, is attached could
be the sink for a group-class stream defined using the group \texttt{gLi}, a 
group of elemental lithium, even though there is no usage of the group
\texttt{gLi} in the conditions block \texttt{example\_block}. A group-class
stream's usage of a group only stipulates what proportions the mass that
stream carries must have. If the mass the stream carries originates from a 
material the isotopes which compose the stream's load may come from any group
or no group at all inside of that material. If the mass the stream carries
ends in a material the isotopes entering the material may go into any group into
which they fit the recipe, or no group at all if the isotope is not a
``controlled" isotope - a topic discussed in section \ref{ssec:control}.

Again, the groups defined using the \texttt{grp} keyword are only recipes. A
given recipe can be used to define multiple structures. Streams all create
their own copy whereas range and ratio restrictions in the same material refer
to the same group in a material's recipe list. As an example, in the input
below, there exist four digital versions of the group \texttt{gFLiBe}, the
original \texttt{grp} recipe, the group which has been added to the material
\texttt{FuelSalt}'s recipe list, and each group created for each stream; and
while these two streams represent two distinct stream objects, the optimization
line seen in the conditions block will minimize the \textit{sum} of the two
streams because they have each been defined with a group of the same name.

\begin{li}
grp gFLiBe
Li  71.7
Be  16
F   103.7

conditions example_block
rng gFLiBe min 0.2 max 0.8
rto gFLiBe val 4 grp2 gUF4
opt dir min type spec_stream gFLiBe

mat FuelSalt -2.805 vol 1 burn 0 ader cnd example_block
3006.06c    0.00028
...

stream to FuelSalt type feed form cont group gFLiBe
stream to FuelSalt type reac form disc group gFLiBe
\end{li}

\subsection{Quick Reference}\label{ssec:group_qr}

\begin{itemize}
    \item{The proportions of elements within a group are normalized to unity.}
    \item{The proportions of isotopes within an element are normalized to unity.}
    \item{Only isotopes belonging to the same element may be listed as a
            component isotope of that element.}
    \item{General Input Structure where \texttt{Element\_x} must be the
            alphabetic periodic table designation for the element, and
            \texttt{iso\_x} must be the alphabetic periodic table
            designation for the parent element followed by a dash
            and then the atomic number of the isotope in question. The value
            which follows a group name in method D of defining a group may
            be any value greater than 0; however, any value other than 1 
            could generate a non-physical answer for a given simulation.
\begin{lt}
grp [Name] <sum> [{Element_1; group_1}] [{Element_1_frac; 1}] ...
<isos> (num_isos) (iso_1) (iso_1_frac) <iso_n> (iso_n_frac) ...
<{Element_n; group_n}> [{Element_n_frac; 1}] ...
<isos> (num_isos) (iso_1) (iso_1_frac) <iso_n> (iso_n_frac) ...
\end{lt}
        }
\end{itemize}
