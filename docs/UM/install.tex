Installing ADER is a five step process: Downloading ADER covered in section
\ref{ssec:dader}, installing the CLP libraries covered
in section \ref{ssec:dclp}, editing the necessary lines in the SERPENT2
base code covered in section \ref{ssec:edit}, compiling the code covered
in section \ref{ssec:compile}, and testing the code, already covered in
section \ref{sec:test}. These instructions assume the user is operating
inside of a linux-like environment.

\subsection{Downloading ADER}\label{ssec:dader}
ADER is available as a public repository hosted at\ldots
\begin{lstlisting}
some_web_address
\end{lstlisting}

Download ADER using your preferred client. The directories contained in the ADER
parent directory, call this folder \texttt{ADER\_Dir}, are\ldots

\begin{itemize}
\item{\texttt{docs} - where this user manual and the API can be found in the
subdirectories \texttt{docs/UM} and \texttt{docs/API}, respectively.}
\item{\texttt{inputs} - where in the subdirectory, \texttt{inputs/Test\_Input},
the file \texttt{test\_input.txt} can be found.}
\item{\texttt{src} - where all of ADER's source files can be found.}
\item{\texttt{System\_Testing} - where all of ADER's system tests can be found
in their respective subdirectories titled after the test which they contain.
Inside of each subdirectory is the test input file and a README file explaining
the operation of the test.}
\end{itemize}

\subsection{Downloading and Installing CLP}\label{ssec:dclp}
CLP is available as a public repository hosted at\ldots

\begin{lstlisting}
https://github.com/coin-or/Clp
\end{lstlisting}

Download CLP using your preferred client. Inside this directory, call it
\texttt{Clp\_Dir}, there is one important subdirectory, \texttt{Clp\_Dir/Clp}.
Inside of this subdirectory executing the following commands in a linux-like
environment should install the Clp libraries on your system\ldots

\begin{li}
./configure
./install-sh
make all
\end{li}

Copy the file \texttt{Clp\_C\_Interface.h}, found in \texttt{Clp\_Dir/Clp/src},
 into your SERPENT2 build directory. 
After this process is complete do not forget to add the path to this
subdirectory to all applicable system paths, most likely just your system
\texttt{PATH}.


\subsection{Editing SERPENT2}\label{ssec:edit}
These instructions assume that the base version of SERPENT2 being modified
is version 2.1.31. ADER DOES NOT WORK WITH VERSIONS OF SERPENT2 EARLIER THAN
2.1.30.
Compatibility with future versions of SERPENT2 and installation directions
are not guaranteed.
To configure SERPENT2 to interact with ADER three steps are necessary. First,
copy all of the files found in \texttt{ADER\_Dir/src}, to the SERPENT2
\texttt{src} directory ( or wherever you build SERPENT2).
This might look something like the following\ldots

\begin{li}
cd ~/SERPENT2_Dir
cp ~/ADER_Dir/src/* ./src/
\end{li}

In the directory \texttt{ADER\_Dir/docs/UM} there is a file named 
``Source\_mod.txt". ``Ln" is short for ``line number".
The contents of this file have the following format\ldots

\begin{lstlisting}
[function_name].c: Line number or description of location in file
{
#ADER MOD BEGIN#
contents to add to file at the designated location
#ADER MOD END#
}
\end{lstlisting}

For each listing, add the code contents found between the
\texttt{ADER MOD BEGIN} and \texttt{ADER MOD END} tags to the
actual SERPENT2 source file given at the location
described: this is either inserting new code lines following an existing
SERPENT2 source line given as either an insertion listing or as a single line
number or as a range of line numbers in which case this range is to be fully
replaced by the ADER code sample. 
A suggestion is to execute these file modifications
from the modifications closest to the end of the file to the
modification closest to the beginning of the file - in this way
the line numbers you look for are unchanged by the addition of
the ADER code.
It is considered good practice to include the tags themselves
,\texttt{ADER MOD BEGIN} and \texttt{ADER MOD END}, 
but this is not strictly necessary.
Consider the file, ``main.c", seen below\ldots

\begin{lstlisting}
#include header.h

void main(args)
{
long j;
long i = 0;

i = 1;

return(i);
}
\end{lstlisting}

Now consider that one of the entries in the file
``Source\_mod.txt" looks like the below\ldots

\begin{lstlisting}
Main: Ln 6-8
{
#ADER MOD BEGIN#
for(j=0; j<10; j++)
{
    i++;
}
#ADER MOD END#
}
\end{lstlisting}

The correctly modified main.c file would look like the below where lines 6 
through 8 of the base file were replaced with the content between and including
the ADER mod tags: ``\texttt{ADER MOD BEGIN}" and ``\texttt{ADER MOD END}".
Please use your best discretion when editing these files. If a line number
direction would cut off a \texttt{for} loop and cause compilation or logic
errors, obviously that line number is incorrect and should be slightly
different.

\begin{lstlisting}
#include header.h

void main(args)
{
long i = 0;
#ADER MOD BEGIN#
for(j=0; j<10; j++)
{
    i++;
}
#ADER MOD END#
return(i);
}
\end{lstlisting}

\subsubsection{Modifying the Makefile}\label{sssec:makefile}
Add the following lines to the SERPENT2 Makefile before the \texttt{OBJS} list
but replace \texttt{[absolute/path/to/the/CLP/library]} with the actual path
to the Clp library on your system.

\begin{li}
# ADER MOD BEGIN #
###############################################################################

# Below should be "-L[absolute/path/to/the/CLP/library] -lclpsolver
#     and -L[absolute/path/to/the/CLP/library] -lclp

LDFLAGS += -L/Clp_Dir/Clp/lib -lClpSolver -L/Clp_Dir/Clp/lib -lClp

# Enable ADER_TEST to run the test_input.txt file to execute the 
#     unit and integration test suite

#CFLAGS += -DADER_TEST

# Enable ADER_INT_TEST to ouput the files necessary for integration
#     testing. Many of these files are human readable and useful 
#     for manual debugging

#CFLAGS += -DADER_INT_TEST

# Enable ADER_DIAG to output a copious amount of debugging data

#CFLAGS += -DADER_DIAG

###############################################################################
# ADER MOD END #
\end{li}

In the directory \texttt{ADER\_Dir/docs/UM} there is a file named 
``Object\_list.txt" - add the contents of this file to the SERPENT2 Makefile
\texttt{OBJS} list. You may alphabetize the entires if you care to but it is
not necessary.

In the directory \texttt{ADER\_Dir/docs/UM} there is a file named 
``Build\_list.txt" - add the contents of this file to the SERPENT2 Makefile
build list - the long list of makefile build instructions at the bottom
of the file. You may alphabetize the entries if you care to but it is
not necessary.

\subsection{Compiling ADER}\label{ssec:compile}
To compile ADER with SERPENT2, after having followed the steps in all previous
subsections of this section, give the command below inside of your 
\texttt{SERPENT2\_Dir/src} directory\ldots

\begin{li}
make all
\end{li}

To compile ADER for unit testing, uncomment the line with the phrase
''\texttt{CFLAGS += DADER\_TEST}". When compiled this way the entire code will
work for no other purpose than ADER unit testing conducted with the file
``\texttt{test\_input.txt}" found inside the 
``\texttt{ADER\_Dir/inputs/Test\_Input}" directory.

The compilation flags ``\texttt{DADER\_DIAG}" and ``\texttt{DADER\_INT\_TEST}"
are covered in section \ref{ssec:output_dev}.
