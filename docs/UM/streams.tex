Streams are the structures by which ADER moves mass into, out of, and between
SERPENT2 materials. Streams are defined as seen below\ldots

\begin{lt}
stream <to> (name_of_sink_material) <from> (name_of_source_material)
    [type] [{feed; reac; redox; remv}] [{group; rem}] [name] 
    [form] [{cont; disc; prop}] (frac) (frac_value)
\end{lt}

The \texttt{to} and \texttt{from} entries specify the stream's source and
sink for the mass which the stream moves. Sources and sinks which are given 
must be valid SERPENT2 materials. Streams do not require \textit{both} a source
and a sink, but rather, at least \textit{one} of either a source or sink. In
the case of a missing source the mass which is brought into the sink material
is assumed to 
come from an infinite, non-reactive (chemically and neutronically) supply
existing outside the simulation bounds. In the case of a missing sink the
mass which is removed from the source material is assumed to disappear - it
leaves the simulation boundary. 

Streams which have elements with unspecified isotopic compositions take these
element's isotopic compositions to be equal to that of the elements in the 
source material at the beginning of the burnup step. These proportions are
updated on each ADER criticality search to account for the changes possibly
induced by discrete form streams. Streams which have elements
with unspecified isotopic compositions and a valid SERPENT2 material as a sink
must have a valid SEPRENT2 material as a source.

Material clusters are collections of SERPENT2 materials which are connected
via streams. If a stream connects two materials, having a valid SERPENT2
material for each source and sink, these two materials become part of the same
material cluster. Because materials may have streams coming from and going to
many other materials, not all materials in a material cluster will be directly
connected by streams; rather, some materials in a material cluster may happen to only
be connected to one another through a series of intermediate streams and
materials. Materials in the same cluster share an optimization solution, that
is, their compositions are optimized collectively according to the singular
optimization target provided, discussed in section \ref{ssec:optimization}. 
Of less impact to the user, material clusters also share a collective
burnup solution though each material may have its own flux and cross sections.
Materials in the same cluster must have the same initial isotopes listed in
their definitions whether or not these isotopes exist in each material at the
beginning of the simulation. If an isotope needs to be listed in a material for
which its concentration is zero, simply include the isotope in the material's
definition with a zero value next to its ZAI identifier.

The \texttt{type} entry has no effect on stream behavior or implementation.
Rather, it serves only as a tag for the \texttt{opt} entries ( discussed
in section \ref{ssec:optimization} ) and as a possible organizational tool
for the user.

The following choice of keywords, \texttt{group} and \texttt{rem}, highlights
the most important distinction between streams; the difference between
group-class (\texttt{group}) and table-class (\texttt{rem}) streams.
While the details of this distinction will be covered in depth in the following
subsections it is sufficient to say, for now, that group-class streams move mass
which follows the recipe outlined by a group while table-class streams move
mass of which the proportions are determined using a table, a \texttt{removal}
table (covered in section \ref{ssec:rem}). Last but not least, an
additional distinction is that group-class streams are variables, the amount of
mass which they move is determined by solving the material optimization problem,
discussed in section \ref{ssec:theory_opt} while table-class streams move a
fixed amount of mass according to user input. Group-class streams have the name
of a group following the keyword \texttt{group} while table-class streams have
the name of a removal table following the keyword \texttt{rem}.
\textbf{Table-class streams can be used to feed mass into a material. The use
of the keyword \texttt{removal} is a legacy-term and does not explicitly mean 
\textit``{removal}" in the sense of the English word}.

The \texttt{form} keyword and its associated values, \texttt{cont},
\texttt{disc} and \texttt{prop}, designate how a stream behaves in time. The
optimization solution, as discussed in section \ref{ssec:theory_opt}, determines
the total mass transfers from the group-class streams interfacing with a material
cluster needed to bring the materials in the material cluster to their optimal
composition. How this mass is moved over time is up to the user through the use
of the \texttt{form} keyword. Streams designated as \texttt{cont} are
``continuous" streams. These streams move mass throughout a burnup step at a
constant rate per second. Streams designated as \texttt{disc} are ``discrete"
streams. These streams move mass as an instantaneous input happening before a
burnup step begins. Streams designated as \texttt{prop} type streams are
``proportional" streams and their method of mass transfer is discussed in
greater detail in the following sections, \ref{sssec:pgroup} and 
\ref{sssec:prem}.

The \texttt{frac} keyword and its associated value are discussed in sections
\ref{ssec:group} and \ref{ssec:rem}.

An important note, mentioned in several spots throughout this manual, is that
only ``disc" form streams impact ADER's reactivity iteration scheme. If ADER
is instructed to iterate on the material composition to match user specified
reactivity targets, ADER will only apply the actions of discrete form streams
on each iteration. Any continuous and proportional streams will not have
their effects on material reactivity incorporated until after the current burnup
step is complete. Additionally, any given material constraint may be out of
bounds during some point in a burnup step if streams of mixed type are used
together as the optimal conditions for the material rely on the full impact
of all streams. As a closing reminder, ADER \textit{does not} account for
nuclear burnup in the \textit{current} burnup step. The effects of nuclear
burnup on composition are assessed by ADER at the beginning of the next
burnup step - i.e. ADER does not predict the effects of nuclear burnup.

\subsection{Group-class streams}\label{ssec:group}

Group-class streams, those streams using the keyword \texttt{group},
operate differently from table-class streams. Group-class
streams are options, given by the user to ADER, for moving mass. As discussed
in section \ref{sec:theory} the amount of mass moved by group-class streams
is determined by the optimization solution. That is, the content of group-class
streams is determined so that the optimal material composition for the material
cluster may be realized.

The mass which a group-class stream moves is described by the
group named after the \texttt{group} keyword. This named group describes the
proportions the mass the stream carries must have. The amount of mass is
determined by the optimization solution. The optimization solution operates on
a volume normalized basis. As such, take the initial units of any solution for
a stream value to be\footnote{Developers note: The actual units of stream return
 values in the code are $\frac{\text{\texttt{atoms}}}{cm \cdot barn}$}
 $\frac{\text{\texttt{atoms}}}{cm^{3}}$ where
``\texttt{atom}" represents a unit of the group-class stream. This unit
is composed of fractional isotopes in accordance with the recipe of the 
group-class stream. 
Take this value to be denoted $s_{n:x}$ where $n$ denotes different
streams and $x$ takes on either $c$, $d$, or $p$ representing continuous,
discrete, and proportional form streams.

Group-class streams do not make use of the \texttt{frac} keyword and as such
their input template appears as follows\ldots

\begin{lt}
stream <to> (name_of_sink_material) <from> (name_of_source_material)
    [type] [{feed; reac; redox; remv}] [group] [name] 
    [form] [{cont; disc; prop}] 
\end{lt}

\subsubsection{Continuous group-class streams}\label{sssec:cgroup}
Continuous group-class streams move an equal amount of mass per unit time over
the length of a burnup step. Taking $\Delta t$ to be the total number of seconds
in the current burnup step continuous group-class streams deliver or remove 
$c_{n} \frac{\text{\texttt{atoms}}}{cm^{3}s}$ to their sink or source material's
respectively where $c_{n}$ is defined in equation \ref{eq:cont_value}.

\begin{equation}\label{eq:cont_value}
c_{n} = \frac{s_{n:c}}{\Delta t}
\end{equation} 

As an example consider below the continuous group-class stream formed using
the \texttt{gFLiBe} group from section \ref{sec:groups} for which there is
no source and for which the sink is the material \texttt{FuelSalt}.

\begin{li}
stream to FuelSalt group gFLiBe form cont type feed 
\end{li}

\subsubsection{Discrete group-class streams}\label{sssec:dgroup}
Discrete group-class streams move their entire mass load, $s_{n:d}$, 
collectively
and instantaneously following the optimization solution and before the 
transport sweep preceding the burnup calculation. With further detail
available in section \ref{sec:theory} discrete group-class streams have a 
unique aspect different from all other streams. Every burnup step ADER
iterates over the optimization solution and a transport sweep until 
the system analog neutron multiplication factor, $k_{eff}^{analog}$, is within
user defined bounds. Every time an optimization solution is arrived at, if
there are discrete streams, these streams will have some value $s_{k,n:d}$ 
where
$k$ denotes the current iteration over the optimization solution for the current
burnup step. The changes in material composition induced by these discrete
stream flows, $s_{k,n:d}$, are applied before the next transport sweep meaning
that the system compositions reflect the actions of discrete group-class
streams. Should another solution of the optimization problem, for the same 
burnup step, be required, the changes induced by these discrete group-class
stream flows, $s_{k,n:d}$, are not undone. Rather, the final $s_{n:d}$ value
reported for discrete group-class streams per burnup step is given by
equation \ref{eq:disc_s}.

\begin{equation}\label{eq:disc_s}
s_{n:d} = \sum_{k}^{K} s_{k,n:d}
\end{equation}

As an example of a discrete group-class stream consider the stream below
formed using the group \texttt{gUF4}, again, with no source but the material
\texttt{FuelSalt} as the sink.

\begin{li}
stream to FuelSalt group gUF4 form disc type reac
\end{li}

\subsubsection{Proportional group-class streams}\label{sssec:pgroup}
Proportional group-class streams attempt to move mass in proportion to the mass
already present. Proportional group-class streams approximate a decay-like
proportional constant, $\lambda_{i,n:p}$, which in a system with no other
effects acting on isotope concentration, would lead to the desired change
in the isotope's, $i$, concentration. 
This is almost never the case in an ADER
simulation and as such the realized change in an isotope's concentration will
likely be different from the desired amount of change. As such, the values
reported in the output by ADER for all proportional streams, group-class or
table-class, are calculated exactly during the burnup solution such that they
reflect the actual amount of the isotope moved, not the pre-calculated amount
which is likely different. Regardless, considering the role that group-class
streams tend to play in ADER simulations the usage of proportional group-class
streams should be undertaken with great caution and a thorough understanding
of how the simulation parameters will interact to affect the proportional
group-class stream constants. Proportional group-class streams may not be
formed using summation groups.

For a proportional stream removing a quantity, $s_{n:p}$, from a material, $m$,
the isotopic proportional constants, $\lambda_{i,n:g}$ for isotope $i$ as
modified by group stream $n$ are calculated, before the burnup step
and \textit{after} the application of any discrete stream effects ( such that
the proportional constants are calculated from the updated material composition
values ), as seen below in equation \ref{eq:prop_remv} where
$f_{i,n:p}$ is the proportion of the stream $n$ accounted for by isotope
$i$, $\rho_{m,k=0}$ is the source material atomic density at the beginning
of the zeroth iteration before any discrete stream actions have been applied to
any materials, $N_{i}$ is the current atomic density of isotope $i$, and
$\Delta t$ is the total time of the current burnup step. In the case of a
desired 100\% removal of an isotope ADER approximates the term
$(s_{n:p} f_{i,n:p} \rho_{m,k=0} N_{i}^{-1})$ as $(1 - 10^{-8})$.

\begin{equation}\label{eq:prop_remv}
\lambda_{i,n:g} = ln(1 - s_{n:p} f_{i,n:p} \rho_{m,k=0} N_{i}^{-1})
    (\Delta t)^{-1}
\end{equation}

For a proportional stream adding a quantity of mass to a material the isotopic
proportional constants are calculated, before the burnup step
and \textit{after} the application of any discrete stream effects ( such that
the proportional constants are calculated from the updated material composition
values ), as seen below in equation \ref{eq:prop_add}.

\begin{equation}\label{eq:prop_add}
\lambda_{i,n:g} = ln(1 + s_{n:p} f_{i,n:p} \rho_{m,k=0} N_{i}^{-1})
    (\Delta t)^{-1}
\end{equation}

During the construction of the burnup matrix, discussed in section
\ref{ssec:burn} these proportional constants are
added to the respective decay constants of the isotopes they seek to modify.

As an example of a proportional group-class stream consider the stream below
formed using the \texttt{gUF4} group and using material \texttt{FuelSalt} as a
source, having no sink.

\begin{li}
stream from FuelSalt group gUF4 form prop type remv
\end{li}

\subsection{Table-class streams}\label{ssec:rem}
Table-class streams, unlike group class streams, are directions from the user to
ADER to move a specific quantity of mass in a specific manner. A stream is
classified as a table-class stream when the keyword \texttt{rem} is used in
the stream definition. This keyword is always followed by the name of 
a \texttt{removal} table. \textbf{The use of the word ``removal" for the name
of this structure is an artifact from earlier code development. These tables
can be used to move mass into, out of, and between SERPENT2 materials. The
materials listed as their \texttt{to} and \texttt{from} entries, sink and
source respectively, operate as sink and source respectively. The name of this
structure \texttt{removal} has no relation to the English word ``removal" other
than spelling.} Removal tables are defined elsewhere in the code using the
format below where elements are entered using their alphabetic periodic table
designation, so \texttt{U} for uranium, while isotopes are entered as elements
followed by \texttt{-A} where A is the atomic number of the isotope; 
\texttt{U-235} for \ce{^{235}U}.

\begin{lt}
removal [Name]
[element_or_isotope_1] [table_value]
...
<element_or_isotope_n> (table_value) 
\end{lt}

Both elements and isotopes may be entered in the same removal table. If a 
\texttt{table\_value} ,$q$, is entered for both an element and one or more of 
its constituent isotopes, the isotopes for which a \texttt{table\_value} was 
entered will retain their corresponding value rather than be overwritten by 
their parent element's value - i.e. an elemental \texttt{table\_value} is a 
default
for all child isotopes of that parent element but this default is overwritten
by any entry for a child isotope. All $q$ values must be greater than or equal
to zero.

As an example, consider the removal table seen below. In this removal table,
\texttt{rem\_table}, all isotopes, $i$, of uranium have an associated 
$q_{i}$ value of 0.1, except for \ce{^{233}U} which has a $q_{i}$ value of zero.

\begin{li}
removal rem_table
U     0.1
U-233 0.0
\end{li}

A single removal table may be used to define multiple streams, each of which
will implement the removal table in its own designated fashion.

All table-class streams require the use of the keyword \texttt{frac} and the
input of its associated value which must be equal to or greater than zero. 
As such the input template for table-class streams appears as such\ldots

\begin{lt}
stream <to> (name_of_sink_material) <from> (name_of_source_material)
    [type] [{feed; reac; redox; remv}] [rem] [name] 
    [form] [{cont; disc; prop}] [frac] [frac_value]
\end{lt}

\textbf{In the following subsections many equations and parameters will depend on a
material value such as density or volume. All material values in reference to
table-class streams are taken at the source material if there is one. If there
is no source material these material values are taken at the sink material.}

\subsubsection{Continuous table-class streams}\label{sssec:crem}
Continuous table-class streams move $s_{n:c}$ amount of material every burnup
step where $s_{n:c}$ is defined by equation \ref{eq:cont_table} where
$q_{i,n:c}$ is the \texttt{table\_value} given for isotope $i$ by the removal
table used to create stream $n$, $N_{i}$ is the number density for that isotope 
(taken after the effects of discrete streams have been applied), $r_{n}$ 
is the \texttt{frac\_value} given for stream $n$, and $V_{m}$ is the material
volume. 

\begin{equation}\label{eq:cont_table}
s_{n:c} = V_{m} \sum_{i}^{I} q_{i,n:c} r_{n} N_{i}
\end{equation}

From equation \ref{eq:cont_table} it is clear that the product of a value from
a removal table and the value given for the stream's \texttt{frac} entry is the
percentage, by atomic density, of that removal table entry to move during 
each burnup step. 
Continuous table-class streams move this mass evenly over time. The transfer
rate, $h_{i} \frac{\text{\texttt{atoms}}}{s^{-1}}$ for isotope $i$ is given by 
equation \ref{eq:cont_iso} where $\Delta t$ is the length of the current burnup 
step in seconds.

\begin{equation}\label{eq:cont_iso}
h_{i} = V_{m} q_{i,n:c} r_{n} N_{i} \Delta t^{-1}
\end{equation}

It is important to remember that isotopes derive their $q_{i,n}$ values from
their parent element's $q_{e}$ value given in the removal table used
to define stream $n$ unless the isotope in question was given its own
$q_{i}$ value in the same removal table.  

As an example of a continuous table-class stream consider the following\ldots

\begin{li}
stream from FuelSalt rem rem_table type redox form cont frac 1000
\end{li}

\subsubsection{Discrete table-class streams}\label{sssec:drem}
Discrete table-class streams move $s_{n:d}$ amount of material every burnup
step where $s_{n:d}$ is defined by equation \ref{eq:disc_table} where
$q_{i,n:d}$ is the \texttt{table\_value} given for isotope $i$ by the removal
table used to create stream $n$, $N_{i}$ is the number density for that isotope 
(taken before the effects of discrete streams have been applied), and $r_{n}$ 
is the \texttt{frac\_value} given for stream $n$.  

\begin{equation}\label{eq:disc_table}
s_{n:d} = V_{m} \sum_{i}^{I} q_{i,n:d} r_{n} N_{i}
\end{equation}

From equation \ref{eq:cont_table} it is clear that the product of a value from
a removal table and the value given for the stream's \texttt{frac} entry is the
percentage of that removal table entry to move during each burnup step. 
Discrete table-class streams move this mass instantaneously at the beginning
of each burnup step. The transfer
amount, $h_{i}$ for isotope $i$ is given 
by equation \ref{eq:disc_iso} where $\Delta t$ is the length of the current burn
up step in seconds.

\begin{equation}\label{eq:disc_iso}
h_{i} = V_{m} q_{i,n:c} r_{n} N_{i} \Delta t^{-1}
\end{equation}

It is important to remember that isotopes derive their $q_{i,n}$ values from
their parent element's $q_{e}$ value given in the removal table used
to define stream $n$ unless the isotope in question was given its own
$q_{i}$ value in the same removal table. With regards to discrete stream
behavior throughout ADER criticality iterations, the effects of discrete
table-class streams are applied on the zeroth iteration for each burnup step,
and no more than that.

As an example of a discrete table-class stream consider the following\ldots

\begin{li}
stream to FuelSalt rem rem_table type feed form disc frac 0.234
\end{li}

\subsubsection{Proportional table-class streams}\label{sssec:prem}
Proportional table-class streams \textit{do not} move a specific quantity of
material in a burnup step. Rather, proportional table-class streams modify the
decay constant of the isotopes in question - this modification may be slight
or may even be of a magnitude to turn a decay constant into a production
constant. The stream-constants, $\lambda_{i,n:t}\frac{1}{s^{-1}}$ for 
isotope $i$ as modified by table-class stream $n$ are determined according to 
equation \ref{eq:stream_cont}.

\begin{equation}\label{eq:stream_cont}
\lambda_{i,n:t} = q_{i,n:p} r_{n} 
\end{equation}

Concerning the inclusion of $\lambda_{i,n:t}$ into the burnup matrix, while this
is covered in detail in section \ref{ssec:burn}, it is sufficient to say for now
that $\lambda_{i,n:t}$ is added to the isotope's decay constant if there
is no source material, this isotope being in the sink material. If there is a
source material $\lambda_{i,n:t}$ is subtracted from the source isotope's decay
constant and this production term is added to the isotope in the sink material
such that the transfer rate is dependent on the source material
isotopic concentration.

As an example of a proportional table-class stream consider the following\ldots

\begin{li}
stream from FuelSalt rem rem_table type remv form prop frac 1.0
\end{li}

\subsection{Quick Reference}
\begin{itemize}
\item{Streams are defined as follows\ldots
\begin{lt}
stream <to> (name_of_sink_material) <from> (name_of_source_material)
    [type] [{feed; reac; redox; remv}] [{group; rem}] [name] 
    [form] [{cont; disc; prop}] (frac) (frac_value)
\end{lt}
    }
\item{Removal tables are defined as follows\ldots
\begin{lt}
removal [Name]
[element_or_isotope_1] [table_value]
...
<element_or_isotope_n> (table_value) 
\end{lt}
    }
\item{Streams require a source or a sink. They may have both, but both are not
required.}
\item{Streams with elements lacking an isotopic composition derive the
element's isotopic composition from the source material.}
\item{Streams with elements lacking an isotopic composition AND with a valid
SERPENT2 material as a sink MUST have a valid SERPENT2 material as a source.}
\item{Material clusters are collections of materials which are connected by
ADER streams}
\item{Group-class streams change their delivered mass each burnup step
according to the optimization solution.}
\item{Table-class streams move user-defined quantities of mass each burnup
step}
\item{Proportional group-class streams may not be formed using summation 
    groups.}
\end{itemize}
