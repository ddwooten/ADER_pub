ADER output is located in the ``[\texttt{input\_file\_name}]\_dep.m" output file
which SERPENT2 produces. Every material under ADER control with groups assigned
via a conditions block or streams, will have output in this file. 

\subsection{Groups}\label{ssec:output_groups}
The 
fraction of a material's atomic density accounted for by each group is printed
out in a vector with the name of ``\texttt{MAT\_m\_GRP\_g\_FRAC}" where 
\texttt{m} is the material name and \texttt{g} is the group name for a group 
which is a member of a material's options list - see section \ref{sec:groups}
for a description of this list. The value at index $j$ of such a vector
corresponds to the value at the end of burnup step $j$. The values displayed
are normalized to the material's atomic density at the beginning of the 
burnup step. 

For example, consider if the group \texttt{gFLiBe} accounted for 60\% of the
material \texttt{FuelSalt}'s atomic density during each of two burnup steps.
The corresponding output line in the ``\_dep.m" file would appear as below\ldots

\begin{li}
MAT_FuelSalt_GRP_gFLiBe_FRAC = [
4.00000E-01, 4.00000E-01 ];
\end{li} 

Summation groups and their component groups are all reported, in no particular
order in the output.

\subsection{Streams}\label{ssec:output_streams}
The mass moved by a stream on each burnup step $j$ is given in vector form
where the vector is named ``\texttt{MAT\_m\_STREAM\_s\_STEP\_AMT}" where 
\texttt{m} is the material name and \texttt{s} is the name of the group or
removal table used to define the group-class or table-class stream in question,
respectively. Results for streams with both a valid sink and source are 
reported by each the sink and the source.
Each vector index $j$ corresponds to burnup step $j$. The units
for stream output are $\frac{\text{\texttt{atoms}}}{barn \cdot cm}$ where the 
value represents
the stream's impact on a per unit volume basis with the normalizing volume being
the material volume for which the stream is reported. This means that if a given
stream moves mass between materials of differing volumes, each material will
report a different value for that stream for each burnup step because their
volumes are different - all thanks to the conservation of mass. In the same
context of \texttt{atoms} from section \ref{sec:groups}, \texttt{atoms} in this
sense refers to a fractional mix of isotopes corresponding to the recipe of the
stream in question. For group-class streams these ``atomic" units will have
the proportions that the group recipe does - this is not necessarily the case
for proportional group-class streams which have a bevy of cautions regarding
their use presented in section \ref{sssec:pgroup}. For table-class streams these
``atomic" units take the proportional make up of the table used to define the
stream except in the case of proportional table-class streams in which case 
the ``atoms" reported is just that, total atoms. The exact composition of
such a stream is not given.
Proportional
type streams do report a true value, in the sense that the number of atoms is
correct, in that the value accounts for nuclear depletion effects.
This value is calculated inside the burnup calculation as what is sometimes
referred to as a ``ghost" nuclide - a false nuclide who's only impact on the
solution is to track and measure some parameter of interest, in this case the
true amount of mass moved by a proportional stream, group-class or table-class.

As an example, consider if stream \texttt{gUF4} had moved 
$0.001 \frac{\text{\texttt{atoms}}}{barn \cdot cm }$ per unit volume into 
material \texttt{FuelSalt} on each of two burnup steps - so  
$0.0002 \frac{\text{\texttt{atoms}}}{barn \cdot cm }$ of uranium and 
$0.0008 \frac{\text{\texttt{atoms}}}{barn \cdot cm }$ of fluorine. 
The corresponding output line in the ``\_dep.m" file would appear as below\ldots

\begin{li}
MAT_FuelSalt_STREAM_gUF4_STEP_AMT = [
1.000000E-02, 1.00000E-02 ];
\end{li} 

Summation streams and their component streams are all reported in the output
in no particular order.

\subsection{Quick Reference}\label{ssec:output_qr}
\begin{itemize}
\item{Group fractions of a material's atomic density, normalized to the
material's atomic density at the beginning of the burnup step, are reported in
a vector with the name of ``\texttt{MAT\_m\_GRP\_g\_FRAC}" where 
\texttt{m} is the material name and \texttt{g} is the group name.}
\item{Stream mass transfers are reported on a per unit volume basis of the
associated material and are reported in 
the vector named ``\texttt{MAT\_m\_STREAM\_s\_STEP\_AMT}" where \texttt{m}
is the source or sink material and \texttt{s} is the name of the group or 
removal table used to define the stream. The units are 
 $\frac{atoms}{barn \cdot cm}$. }
\end{itemize} 

\subsection{Developers Output}\label{ssec:output_dev}
In the advanced compilation options for ADER, \texttt{ADER\_DIAG} and 
\texttt{ADER\_INT\_TEST}, numerous additional outputs are produced. Any
person who is making use of these outputs is expected to be familiar with the
documentation of ADER as well as the source code. As such, only a brief summary
of the additional files is given for each option. A more thorough understanding
may be had by inspection of the functions which produce these outputs.

\subsubsection{ADER\_INT\_TEST}\label{ssec:output_dev_int}
Compiling SERPENT2 and ADER with the ``\texttt{-DADER\_INT\_TEST}" flag will
cause the following outputs to be produced. In the file name templates given
\texttt{m} stands for a material name and \texttt{n} is a material cluster 
number.

\begin{itemize}
\item{``ADER\_Clp\_Model\_Material\_\texttt{m}\_Conformity.test" - contains an
element by element comparison of the simplex model in ADER memory with the
simplex model in CLP memory.}
\item{``ADER\_Cluster\_Composition\_Matrices.json" - contains all the
information needed to construct the simplex problem in a json format.} 
\item{``Cluster\_\texttt{n}\_Material\_Composition\_Matrix.csv" - contains every
entry of the simplex matrix for material cluster \texttt{n}.}
\end{itemize}

\subsubsection{ADER\_DIAG}\label{sssec:output_dev_diag}
Compiling SERPENT2 and ADER with the ``\texttt{-DADER\_DIAG}" flag will
cause the following outputs to be produced. In the file name templates given
\texttt{m} stands for a material name, \texttt{s} stands for the numeric index
of a given burnup step, \texttt{ss} stands for the numeric index of the sub-step
for burnup step \texttt{s}, \texttt{n} is a material cluster number, and 
\texttt{k} is the index of the criticality search iteration.

\begin{itemize}
\item{``Cluster\_Parent\_m\_Burn\_Matrix\_Step\_\texttt{s}\_Sub\_Step\_\texttt{ss}.csv" - contains a copy of the burnup matrix for the material cluster who's
parent is material \texttt{m} for the given burnup step and sub-step.}
\item{``Cluster\_Parent\_\texttt{m}\_Pred\_Step\_\texttt{s}\_Sub\_Step\_\texttt{ss}\_Matrix\_Comparison.test" - contains a comparison of every compatible entry in
the burnup matrix as generated by SERPENT2 and ADER for the prediction step
of a predictor-corrector burnup simulation.}
\item{``Cluster\_Parent\_\texttt{m}\_Corr\_Step\_\texttt{s}\_Corr\_Sub\_Step\_\texttt{ss}\_Matrix\_Comparison.test" - contains a comparison of every compatible 
entry in
the burnup matrix as generated by SERPENT2 and ADER for the corrector step
of a predictor-corrector burnup simulation.}
\item{``ADER\_Memory\_Lists.test" - contains \texttt{WDB} address information 
for a variety of data.}
\item{``\texttt{m}\_XS\_End\_of\_Step\_\texttt{s}.txt" - contains a comparison 
of the isotopic cross sections (absorption and fission) between SERPENT2 and 
ADER after all criticality search iterations for a given burnup step.}
\item{``\texttt{m}\_XS\_step\_\texttt{s}\_iter\_\texttt{k}.txt" - contains a
comparison of the isotopic cross sections (absorption and fission) between
SERPENT2 and ADER for the given burnup step \texttt{s} and inner criticality
iteration \texttt{k}.}
\end{itemize}

