The details of ADER's computations and the theory behind its operation are
laid out in this section.

\subsection{Groups}\label{theory_groups}
Take $F_{e|u}$ to be the input values given for each element, $e$, in a group,
$u$. Take $F_{i|e,u}$ to be the input value given for each isotope, $i$, as
part of element $e$ in group $u$. The normalized equivalents, $f_{e,u}$ and
$f_{i|e,k}$, are defined by equations \ref{eq:t_group_e} and \ref{eq:t_group_i}
respectively. 

\begin{equation}\label{eq:t_group_e}
f_{e,u} = \frac{F_{e,u}}{\sum_{e}^{E} F_{e,u}}
\end{equation}

\begin{equation}\label{eq:t_group_i}
f_{i|e,u} = \frac{F_{i|e,u}}{\sum_{i|e}^{I|e} F_{i|e,u}}
\end{equation}

Concerning the elements in groups which are not given an explicit isotopic
composition, referred to as ``unfixed" elements - calculations in SERPENT2 are 
ultimately done on an isotopic basis
and so all elements must be composed of isotopes. Groups which are a part of a
material's options list, see section \ref{sec:groups} for an explanation of this
term, derive the isotopic compositions for their unfixed elements using the 
isotopic composition of the same element in the host material. These
$f_{i:e,u}$ values are derived as seen in equation \ref{eq:t_groups_unfixed}
where $N_{i:e}$ is the number density of isotope $i$ of element $e$ in the 
material of interest.

\begin{equation}\label{eq:t_groups_unfixed}
f_{i|e,u} = \frac{N_{i|e}}{\sum_{i|e}^{I|e} N_{i|e}}
\end{equation}

Take $f_{i,u}$ to be the normalized fraction of isotope $i$ in group $k$. These
values are derived as seen in equation \ref{eq:t_group_I}.

\begin{equation}\label{eq:t_group_I}
f_{i,u} = f_{e,u}f_{i|e,u}
\end{equation}

Take $g_{u}$ to be the normalized fraction of a material's atomic density
which is taken to ascribe to the recipe for group $u$ - that is, the isotopes
composing this subset of the material density all have a normalized abundance of
$f_{i,u}$ within the subset. For instance, consider the atomic density and 
makeup of the material \texttt{FuelSalt} from section \ref{sec:intro} and
consider the group \texttt{gFLiBe} from section \ref{sec:groups}. The $g_{u}$
value for \texttt{gFLiBe} in the material \texttt{FuelSalt} could be said
to be anywhere from zero to ~0.76.

Take $E_{j}$ to be the number density of element $j$ in a material. Take
$I_{k}$ to be the atomic density of isotope $i$ in a material. Take
$\mu_{j}$ to be the portion of element $j$'s number density not attributed to
group structures. Take $\mu_{k}$ to be the portion of isotope $k$'s atomic
density which is not attributed to group structures. In the case of a
``controlled" element or isotope, a concept covered in section 
\ref{ssec:control}, the $\mu$ value would be zero. The relationships then
seen in equations \ref{eq:t_groups_e_balance} and \ref{eq:t_groups_i_balance}
may be established.

\begin{equation}\label{eq:t_groups_e_balance}
E_{j} = \mu_{j} + \sum_{u}^{U} g_{u}f_{e,u} 
\end{equation}

\begin{equation}\label{eq:t_groups_i_balance}
I_{k} = \mu_{k} + \sum_{u}^{U} g_{u}f_{i,u} 
\end{equation}

In section \ref{ssec:ratios} the concept of a relative abundance constraint
between two groups was developed. Equation \ref{eq:rto_mb} can be 
expressed linearly as two inequalities as
shown in equations \ref{eq:rto_min} and \ref{eq:rto_max} where $r_{m}$ is the
input \texttt{min\_value} and $r_{M}$ is the input \texttt{max\_value}.

\begin{equation}
\label{eq:rto_min}
-\infty \leq -g_{1} + r_{m}g_{2} \leq 0
\end{equation}

\begin{equation}
\label{eq:rto_max}
0 \leq -g_{1} + r_{M}g_{2} \leq \infty
\end{equation}

Summation groups, those groups defined according to method D in section 
\ref{sec:groups}, have their $g_{k}$ values - denoted $g_{k:Y}$ - determined 
according to equation \ref{eq:t_group_sum} where $g_{k|Y}$ is the $g_{k}$ 
value for a group used in the definition of summation group $Y$ and $w_{k|Y}$
is the weight value entered for group $k$ used in the definition of summation
group $Y$. In section \ref{sec:groups} this value is shown as a mandatory input
of ``\texttt{1}" though in reality it may take any value greater than 0 - but
any value other than 1 may lead to a non-physical solution in the optimization
problem. 

\begin{equation}\label{eq:t_group_sum}
g_{k:Y} = \sum_{k|Y}^{K|Y} g_{k|Y} w_{k|Y}
\end{equation}

\subsection{Group-class Streams}\label{ssec:t_streams}
For group-class streams take $s_{v}$ to be the atomic density of this mass
load where an ``atomic" unit is composed of isotopes in proportion to the
$f_{k,u}$ values of the group, which are denoted $f_{k,v}$ for groups which
are used to form streams. 

Take $E_{j,\Delta}$ to be the number density of element $j$ in a given stream's
mass load. Take $I_{k,\Delta}$ to be the atomic density of isotope $i$ in a 
given stream's mass load. The relationships then seen in equations 
\ref{eq:t_streams_e_balance} and \ref{eq:t_streams_i_balance} may be 
established.

\begin{equation}\label{eq:t_streams_e_balance}
E_{j,\Delta} = s_{v}f_{e,v} 
\end{equation}

\begin{equation}\label{eq:t_streams_i_balance}
I_{k,\Delta} = s_{v}f_{i,v} 
\end{equation}

Concerning the elements in streams which are not given an explicit isotopic
composition, referred to as ``unfixed" elements - calculations in SERPENT2 are 
ultimately done on an isotopic basis
and so all elements must be composed of isotopes. Streams 
derive the isotopic compositions for their unfixed elements using the 
isotopic composition of the same element in the source material. These
$f_{i|e,v}$ values are derived as seen in equation \ref{eq:t_streams_unfixed}
where $N_{i|e}$ is the number density of isotope $i$ of element $e$ in the 
source material of interest. Streams with unfixed elements and no SERPENT2
material as a source are not permitted.

\begin{equation}\label{eq:t_streams_unfixed}
f_{i|e,v} = \frac{N_{i|e}}{\sum_{i|e}^{I|e} N_{i|e}}
\end{equation}

Summation streams, group-class streams constructed with summation
groups have their $s_{v}$ values - denoted $s_{v:Y}$ - determined 
according to equation \ref{eq:t_streams_sum} where $s_{v|Y}$ is the $s_{v}$ 
value for a stream used in the definition of summation stream $Y$ and $q_{v|Y}$
is the weight value entered for group $v$ used in the definition of summation
stream $Y$. In section \ref{sec:groups} this value is shown as a mandatory input
of ``\texttt{1}" though in reality it may take any value greater than 0 - but
any value other than 1 may lead to a non-physical solution in the optimization
problem. 

\begin{equation}\label{eq:t_streams_sum}
s_{v:Y} = \sum_{v|Y}^{V|Y} s_{v|Y}q_{v|Y}
\end{equation}

\subsection{Table-class Streams}\label{sssec:t_tables}
The change in an isotope's, $i$, atomic density, $z_{i,v}$, induced by a 
continuous or discrete table-class stream, $v$ is given by equation 
\ref{eq:t_tables_z} where $q_{i,v}$ is the \texttt{table\_value} given for 
isotope $i$ by the removal table used to create stream $v$, $N_{i}$ is the 
number density for that isotope, and $r_{v}$ is the \texttt{frac\_value} given 
for stream $v$.

\begin{equation}\label{eq:t_tables_z}
z_{i,v} = q_{i,v} r_{v} N_{i}
\end{equation}

The change in an isotope's, $i$, atomic density, $z_{i,v}$, induced by a 
proportional table-class stream, $v$ is given by equation 
\ref{eq:t_tables_zp} where $q_{i,v}$ is the \texttt{table\_value} given for 
isotope $i$ by the removal table used to create stream $v$, $N_{i}$ is the 
number density for that isotope, and $r_{v}$ is the \texttt{frac\_value} given 
for stream $v$. If the proportional stream in question, $v$, has only a single
valid SERPENT2 material as a sink, the sign of $q_{i,v}$ is positive. Otherwise,
the sign of $q_{i,v}$ is negative. 

\begin{equation}\label{eq:t_tables_zp}
z_{i,v} = N_{i}e^{q_{i,v} r_{v} \Delta t}
\end{equation}

The total change in an isotope's, $i$, atomic density due to all 
tale-class streams is denoted $r_{i}$ is given by equation
\ref{eq:t_tables_r}. It should be noted, that when this summation involves
values for a proportional stream the $r_{i}$ value becomes an approximation
as the proportional stream will move more or less mass based on nuclear
depletion and the action of other streams.

\begin{equation}\label{eq:t_tables_r}
r_{i} = \sum_{v}^{V} z_{i,v}
\end{equation}

There are of course $z_{e,v}$ and $r_{e}$ equivalents for elements as well.

\subsection{Criticality Control}\label{ssec:t_crit}
A weighted sum over isotopes in a material forms the reactivity constraint that
may be applied. This constraint is derived from the expression for the 
multiplication factor as found in Equation~\ref{eq:reac}:

\begin{equation}
\label{eq:reac}
k_{eff} = P_{NL} \frac{\sum\limits^{M}_{m}\phi_m\omega_m\nu\Sigma_{f}^{m}}
{\sum\limits^{M}_{m}\phi_m\omega_m   \Sigma_{a}^{m}}
\end{equation}

where $\phi_m$, $\omega_m$, $ \nu\Sigma_{f}^{m}$ and $ \Sigma_{a}^{m}$ are, 
respectively, the scalar neutron flux, the volume fraction, the spectrum 
averaged neutron production cross section, and the macroscopic absorption 
cross section for each material $m$. $P_{NL}$ is the neutron non-leakage 
probability. In ADER, the ability to control $k_{eff}$ is limited to the 
case of a single neutron multiplying material; take $\nu\Sigma_{f}=0$ for 
every material but the multiplying material $M$ and as such 
Equation \ref{eq:reac} can be rewritten as follows:

\begin{equation}
\label{eq:reac_one_material}
k_{eff} = P_{NL} \frac{\phi_M\omega_{M}\Sigma_{a}^{M}}{\sum\limits^{M}_{m}\phi_m
\omega_m \Sigma_{a}^{m}} \frac{\nu\Sigma_{f}^{M}}{\Sigma_{a}^{M}}
\end{equation}

The probability of a neutron being absorbed in the multiplying material is 
defined as follows:

\begin{equation}
\label{eq:abs_prob}
P_{A} = P_{NL} \frac{\phi_M\omega_{M}\Sigma_{a}^{M}} {\sum\limits^{M-1}_{m}
\phi_{m}\omega_m\Sigma_{a}^{m}}
\end{equation}

Then $k_{eff}$ can be calculated as:

\begin{equation}
\label{eq:reac_modified}
k_{eff} = P_{A} \frac{\nu\Sigma_{f}^{M}}{\Sigma_{a}^{M}} = P_{A} 
\frac{\sum\limits^{I}_{i}\nu\Sigma_{f}^{i}}{\sum\limits^{I}_{i}\Sigma_{a}^{i}}
\end{equation}

where $\nu\Sigma_{f}^i$, and $\Sigma_{a}^i$ are, respectively, the spectrum 
averaged neutron production cross section, and the absorption cross 
section for every isotope $i$ in the multiplying material $M$. 
This relation is expected to hold for simulations in which there
is a dominant reactive material and for which $\nu\Sigma_{f} \approx 0$ for 
all other materials.

In this case, given lower and upper bounds for the multiplication factor of 
the system, $k_{eff}^{min}$ and $k_{eff}^{max}$ respectively, 
Equation \ref{eq:reac_modified} can be made linear as in Equations 
\ref{eq:k_min} and \ref{eq:k_max}.


\begin{equation}
\label{eq:k_min}
0 \geq \frac{k_{eff}^{min}}{P_{A}} \sum \limits_{k}^{K} \sigma_{a}^{k} I_{k} 
- \sum \limits_{k}^{K} \nu^{k} \sigma_{f}^{k} I_{k}
\end{equation}

\begin{equation}
\label{eq:k_max}
0 \leq \frac{k_{eff}^{max}}{P_{A}} \sum \limits_{k}^{K} \sigma_{a}^{k} I_{k} 
- \sum \limits_{k}^{K} \nu^{k} \sigma_{f}^{k} I_{k}
\end{equation}

A key assumption of this linearization process is that 
$\frac{\partial P_{A}(m...M)}{\partial M} = 0$ when in truth
$P_{A}$ is a function of the composition of material $M$. The impacts of this 
approximation are expected to be quite small but it will affect all 
simulations, more so those with strong leakage effects.

\subsection{Constructing the Optimization Problem}\label{ssec:t_construct}
A discussion of the construction of the optimization problem would not be
complete without a discussion of ADER's interactions with SERPENT2 and a view
of the overall program flow. Figure \ref{fig:flowchart1} provides a simplified
view of ADER's interactions with SERPENT and the overall simulation progression.
In figure \ref{fig:flowchart2} the ``ADER Optimization" box from figure 
\ref{fig:flowchart1} is expanded as a process into its own flowchart.

\begin{figure}\label{fig:flowchart1}
\caption{A simplified schematic of interactions between SERPENT2 and ADER.}
\begin{centering}
\begin{tikzpicture}[node distance = 1.5cm, every text node part/.style={font=\small}]
\node (start) [startstop] {SERPENT2 Starts};
\node (input) [io, below=0.5cm of start] {User input};
\node (dec1) [decision, below=0.5cm of input] {Are there burnup steps remaining?};
\node (first) [process, below=0.5cm of dec1] {Monte-Carlo transport sweep};
\node (end) [startstop, right=0.5cm of dec1] {End};
\node (ader_opt) [process, below=0.5cm of first] {ADER Optimization};
\node (ader_burn) [process, below=0.5cm of ader_opt] {ADER Burnup};

\draw [arrow] (start) -- (input);
\draw [arrow] (input) -- (dec1);
\draw [arrow] (dec1) -- (first);
\draw [arrow] (dec1) -- node[anchor=west] {yes} (first);
\draw [arrow] (dec1) -- (end);
\draw [arrow] (dec1) -- node[anchor=north] {no} (end);
\draw [arrow] (first) -- (ader_opt);
\draw [arrow] (ader_opt) -- (ader_burn);
\draw [arrow] (ader_burn) -- ([shift={(-2cm,0cm)}]ader_burn.west) |- ([shift={(-2cm,0cm)}]dec1.west) -- (dec1);
\end{tikzpicture}
\end{centering}
\end{figure}

\begin{figure}\label{fig:flowchart2}
\caption{A detailed look of the ``ADER Optimization" box from figure
\ref{fig:flowchart1}. This diagram picks up at the entry to the ``ADER
Optimization" box and exits out to the ``ADER Burnup" box also in figure
\ref{fig:flowchart1}.}
\begin{centering}
\begin{tikzpicture}[node distance = 1.5cm, every text node part/.style={font=\small}]
\node (ader_gather) [process] {ADER updates $f_{i,v}$ and $f_{i,u}$ values for 
unfixed elements};
\node (dec1) [decision, below=0.5cm of ader_gather] {Are there any group-class
streams?};
\node (ader_build) [process, below=0.5cm of dec1] {ADER builds the
optimization problem};
\node (clp) [process, below=0.5cm of ader_build] {ADER passes off the the
optimization problem to CLP which returns a vector of solutions};
\node (dec2) [decision, below=0.5cm of clp] {Is this the first criticality 
iteration?};
\node (rem) [process, right=1cm of dec2] {ADER collects $z_{i,v}$ values  and 
applies the effects of any discrete table-class streams to material 
compositions};
\node (ader_apply) [process, below=0.5cm of dec2] {ADER apple's the effects of
any discrete group-class streams to material compositions};
\node (monte) [process, below=0.5cm of ader_apply] {Monte Carlo transport 
sweep};
\node (dec3) [decision, below=0.5cm of monte] {Is iteration count at maximum
OR $k_{eff}^{min} \leq k_{eff}^{analog} \leq k_{efF}^{max}$?};
\node (end) [startstop, right=1cm of dec3] {End};

\draw [arrow] (ader_gather) -- (dec1);
\draw [arrow] (dec1) -- (ader_build);
\draw [arrow] (dec1) -- node[anchor=west] {yes} (ader_build);
\draw [arrow] (dec1) -- node[anchor=north] {no} ([shift={(-2cm,0cm)}]dec1.west) |- ([shift={(-2cm,0cm)}]dec2.west) -- (dec2);
\draw [arrow] (ader_build) -- (clp);
\draw [arrow] (clp) -- (dec2);
\draw [arrow] (dec2) -- (ader_apply);
\draw [arrow] (dec2) -- node[anchor=west] {no} (ader_apply);
\draw [arrow] (dec2) -- (rem);
\draw [arrow] (dec2) -- node[anchor=north] {yes} (rem);
\draw [arrow] (rem) -- (rem.south) |- ([shift={(1cm,0cm)}]ader_apply.east) -- (ader_apply);
\draw [arrow] (ader_apply) -- (monte);
\draw [arrow] (monte) -- (dec3);
\draw [arrow] (dec3) -- (end);
\draw [arrow] (dec3) -- node[anchor=north] {yes} (end);
\draw [arrow] (dec3) -- node[anchor=north] {no} ([shift={(-3cm,0cm)}]dec3.west) |- ([shift={(-2cm,0cm)}]ader_gather.west) -- (ader_gather);
\end{tikzpicture}
\end{centering}
\end{figure}

In a SERPENT2 burnup simulation, following the initial Monte Carlo transport
sweep done at the beginning of every burnup step, ADER enters its criticality
search iterations - even if a criticality search has not been asked for. The
default values for \texttt{kmin} and \texttt{kmax} will cause the criticality
check to pass regardless.

The first consequential action ADER takes in each iteration is to determine
the $f_{i|e,u}$ and $f_{i|e,v}$ values for the isotopes of unfixed elements
in groups and group-class streams. Following this assessment all of the
$z_{i,v}$ values are calculated for table-class streams.

This point marks the first major divergence of any ADER simulation.
\textit{If and only if} the only streams entered by the user are table-class
streams the simulation proceeds to the application of the effects of discrete
form streams. There is no optimization process in this case because the user 
has only given direct orders to ADER in the form of table-class streams, there 
are no choices for ADER to make in the form of group-class streams. 

In the case that at least one group-class stream, attached to a material, exists
in the simulation, the simulation would then proceed to build and solve the
optimization problem for each material cluster.

The CLP library expects a linear programming matrix from ADER.
Figure \ref{fig:opt_matrix}
depicts the scheme for constructing the linear programming matrix. Column bounds
are presented above the appropriate column whereas row bounds are presented to 
the left of the appropriate row. Below the column bounds are the variables which
the columns represent and to the right of the row bounds are the equation
number, if any, of the equation that row is modeled after. 
For the sake of brevity
the matrix in Figure~\ref{fig:opt_matrix} is for one material only though many
materials may be involved in such a matrix should they be linked together
by shared mass transfers. In which case the only variables shared between 
materials in the same material cluster are the
group-class streams and the stream equations they are a part of are the only
coupling equations; aside from transfers by table-class streams but those
are only represented in the linear programming matrix, they are handled by
other routines all together. If a second material were to be included in this
matrix then, perhaps, the stream entries in the fourth, fifth, and sixth 
columns would have non-zero coefficients for some $E_{j}^{d}$ and $I_{k}^{d}$ 
rows of the second material. The coefficients used for the construction of these
matrices are normalized to the atomic density from the beginning of the
optimization process for the host material.

Working down the matrix row by row the first row encountered represents
equation \ref{eq:rto_min} with arbitrary groups $g_{1}$ and $g_{2}$ whereas 
the next row 
down represents equation \ref{eq:rto_max}. The third row, what will be referred
to as an elemental future row represents the atom balance for element $j$ where
$f{e_{j}^{u}}$ is the fractional proportion of element $j$ in group $u$.
The novel column involved here is an elemental future column whose inclusion
in the same row closes the equation where an ``elemental future value",
$E_{j}^{f}$, is the fraction of a material's atomic density ( relative to
the density at the beginning of the step ) that is taken up by element $e_{j}$.
The bounds for this row are those for a
free element, a concept covered in section \ref{ssec:control}, 
those elements which are permitted to have portions
of the element not tied up in declared group structures. In the case of a
controlled element, those who's complete abundance must be accounted for by
group structures, the lower bound is changed to zero. 
The fourth row, an elemental
delta row, represents the change in the abundance of element $j$,
$E_{j}^{d}$, as caused by all group-class streams where
$f{e_{j}^{v}}$ is the fractional proportion of element $j$ in stream $v$. 
Of course the elemental delta column 
is involved to close the balance. The fifth row, or balance row, is what ties
together $E_{j}^{f}$ and $E_{j}^{d}$. The bounds, $\alpha$ and $\beta$,
are equal and represent $E_{j}^{c} + r_{e_{j}}$ constituting an element 
balance ``in time” where $E_{j}^{c}$ represents the present fractional
abundance of element $j$. The fifth row requires, straightforwardly, that the
future amount of an element be equal to the current amount plus any delta,
or change, in the element's abundance. The sixth row is an isotopic balance row
requiring that the abundance of an element be equal to the abundance of its
constituent isotopes. 
The following three rows, the seventh, eighth, and ninth,
are the isotopic versions of the elemental future, delta, and balance rows 
where $f$ values are for the isotopic fractional proportions.
$\gamma$ and $\delta$ are
equal and represent $I_{k}^{c} + r_{i}$ where $I_{k}^{c}$ represents
the present fractional abundance of isotope $k$. In the tenth and eleventh rows
Equations \ref{eq:k_max} and \ref{eq:k_min} find representation with $\eta$
and $\theta$ respectively representing terms of the expanded sum found in the
referenced equations; $\frac{k_{eff}^{min}}{P_{A}} \sigma_{a}^{k} - \nu^{k}
\sigma_{f}^{k}$ and
$\frac{k_{eff}^{max}}{P_{A}} \sigma_{a}^{k} - \nu^{k}
\sigma_{f}^{k}$. 
The twelfth row represents equation \ref{eq:oxi_mb}, accounting for the
contributions of the future quantity of an element to a material's averaged
oxidation state (or however the weighted sum is interpreted).
The thirteenth row, or Pres row, exists
when the user instructs ADER to balance inflows with outflows as discussed
in section \ref{ssec:pres}. The
pres row requires that the net stream transfers in a material come to zero. The
effects of table-class streams are captured in $\upsilon$ and $\omega$ as seen
in equation \ref{eq:upsilon_bound}.
The fourteenth row represents a closure for a group defined with method D from
section \ref{sec:groups}; a summation group. In this case the summation group
is $g_{3}$ which was defined as follows\ldots

\begin{li}
grp g3 sum
g1  w1 
g2  w2
\end{li}

The fifteenth row represents a closure relationship for a summation stream,
a group-class stream built using a summation group. In this case the summation
stream is $s_{3}$.
The final row is the optimization, or Opt row. This row
indicates to the simplex routine which variables to minimize or maximize. In
figure \ref{fig:opt_matrix} the opt row is indicating that $g_{1}$ is the
optimization target.

\begin{equation}\label{eq:upsilon_bound}
\upsilon = \omega = \sum_{i}^{I} r_{i}
\end{equation}

    \begin{equation*}
    \label{fig:opt_matrix}
        \centering
        \begin{blockarray}{cccccccccccc}
                               &                   & [b_{m},b_{M}]  &[0, \infty)
           &[0, \infty)        &[0,\infty)         & [0,\infty)     &[0, \infty)
           &[0, \infty)        &(-\infty,\infty)         & [0,\infty)     &
            (-\infty0,\infty)  \\ 
                               &                   & g_{1}          &g_{2}
            &g_{3}             & s_{1}             & s_{2}          &s_{3}
            &E_{j}^{f}         & E_{j}^{d}         & I_{k}^{f}      &
            I_{k}^{d} \\
                               &                   &                &
            &                  &                   &                &
            &                  &                   &                &
             \\ 
            \begin{block}{cc[cccccccccc]}
             {(-\infty,0]}     & \text{Eq.}\ref{eq:rto_min} & -1    &r_{m}
            &                  &                   &                &
            &                  &                   &                &
             \\
             {[0,\infty)}      & \text{Eq.}\ref{eq:rto_max} & -1    &r_{M}
            &                  &                   &                &
            &                  &                   &                &
             \\
             {(-\infty,0]}     & E_{j}^{f}         & f_{e_{j}^{1}}  &f_{e_{j}^{2}}
            &                  &                   &                   &
            &-1                &                   &                   &
             \\
            {[0,0]}            & E_{j}^{d}         &                   &
            &                  & f_{e_{j}^{1}}     & f_{e_{j}^{2}}     &
            &                  & -1                &                   &
             \\
            {[\alpha, \beta]} 
                               & E_{j}^{b}         &                   &
            &                  &                   &                   &
            &1                 & -1                &                   &
             \\
            {[0,0]}            & E_{j}^{i}         &                   &
            &                  &                   &                   &
            &-1                &                   & 1                 &
             \\
            {(-\infty,0]}      & I_{k}^{f}         & f_{i_{k}^{1}} &f_{i_{k}^{2}}
            &                  &                   &                   &
            &                  &                   & -1                &
             \\
            {[0,0]}            & I_{k}^{d}         &                   &
            &                  & f_{i_{k}^{1}}     & f_{i_{k}^{2}}     &
            &                  &                   &                   &
            -1 \\
            {[\gamma, \delta]}
                               & I_{k}^{b}         &                   &
            &                  &                   &                   &
            &                  &                   & 1                 &
            -1 \\ 
            {[0, \infty)}      & \text{Eq.}\ref{eq:k_max}&             &
            &                  &                   &                   &
            &                  &                   & \eta              &
             \\
            {(-\infty, 0]}     & \text{Eq.}\ref{eq:k_min} &            &
            &                  &                   &                   &
            &                  &                   & \theta            & 
             \\
            {[O_{m},O_{M}]}    & \text{Eq.}\ref{eq:oxi_mb} &           &
            &                  &                   &                   &
            &v_{e}w_{e}        &                   &                   &
             \\
            {[\upsilon,\omega]} & \text{Pres}      &                   &
            &                  & 1                 & 1                 &
            &                  &                   &                   &
             \\
             {[0,0]}           & \text{Eq.}\ref{eq:t_group_sum} & -w_{1|3} &-w_{2|3}
            &1                 &                   &                &
            &                  &                   &                &
             \\
             {[0,0]}           & \text{Eq.}\ref{eq:t_streams_sum} &  &
            &                  &-q_{1|3}           &-q_{2|3}        &1
            &                 &                   &                &
             \\
                               & \text{Opt}        & 1                 &
            &                  &                   &                   &
            &                  &                   &                   &
             \\
            \end{block}
        \end{blockarray}
    \end{equation*}

\subsection{Solving the Optimization Problem}\label{ssec:clp}
Once the matrix seen in figure \ref{fig:opt_matrix} has been constructed by ADER
it is converted into a dense column-major format and passed off to the CLP 
library which solves this matrix as a simplex problem. CLP then returns a
vector containing the atomic density of each group in a material, normalized to
the material's density at the beginning of the optimization process. This
vector also contains all the group-class mass load values needed to bring
the material cluster to the optimal state. If the simulation is in the first
criticality search iteration for a burnup step, the actions of all discrete 
form streams,both group-class and table-class, are applied to materials. If the
simulation is past the first criticality search iteration for the current
burnup step only the actions of discrete group-class streams are applied to
materials. Following these actions a Monte Carlo transport sweep is run with
all the same cycle and batch parameters as specified by the user in the
SERPENT2 input files. At this point, if the system $k_{eff}^{analog}$ is within
the bounds as set by the user (or the default bounds) or if the criticality
search has already used up all of its iterations, the simulation progresses
on to the burnup calculation. If not, another criticality search iteration is
launched starting with the determination of the isotopic composition for
all unfixed elements in groups and group-class streams.

\subsection{Nuclear Burnup Calculations}\label{ssec:burn}
ADER's burnup routine rides along the iteration scheme employed by the SERPENT2
burnup routine. That is to say that ADER is compatible with any burnup
correction scheme that the user employs through SEREPNT2. In truth, the only
affect that a burnup iteration scheme has on ADER is to change the cross
sections used in any criticality control rows in the optimization matrices.
ADER extends the burnup capabilities of SERPENT2 to include the effects
of both group and table-class streams.
The coefficients in the burnup
matrix are those from the Bateman equation as seen in equation
\ref{eq:Bateman} for the one energy group and zero 
dimensionality case, where $N$ is the number density of nuclide $n$,
$t$ is time, $b_{m \to n}$ is the branching ratio for the decay of 
nuclide $m$ into $n$, $\lambda$ is the decay constant
for its sub-scripted nuclide,
$q$ goes over all neutron induced absorption reactions for a given isotope, 
$a_{m \to n}^{q}$ is the branching ratio for isotope $m$ into $n$ due to
reaction $q$, $\sigma_{x}^{y}$ is the effective microscopic
cross section of reaction $x$ for isotope $y$, $\phi$ is the
scalar neutron flux, $d$ denotes all
transmutation reactions for a given isotope, $R_{n}(t)$ is a
fractional removal (or addition) rate for isotope $n$ at
time $t$, and $F_{n}(t)$ is a feed (or removal) amount for isotope
$n$ at time $t$. These last two terms in equation \ref{eq:Bateman}
account for proportional and continuous form streams
respectively.

A highly truncated burnup scheme can be
seen in figure \ref{fig:burn_matrix} in which there are two isotopes,
\ce{^{233}U} and \ce{^{135}Xe}, and two streams; $S_{c}$ representing a
continuous stream with a constant injection rate and $S_{p}$ representing a
proportional stream with a transfer rate dependent upon the concentration
of the substances to be transferred. There are, of course, two matrices as well.
The burnup matrix to the left holding the coefficients of the Bateman equation
and the second, to the right, holding the initial concentrations of isotopes
and the values for the streams. The first column of the first row gives
the creation and destruction of \ce{^{233}U} which is dependant on the
concentration of \ce{^{233}U} with $\Gamma$ representing nuclear destruction as
seen in equation \ref{eq:Gamma_def}. The third column of the first row holds
the fraction of stream $S_{c}$ that \ce{^{233}U} comprises. These entries
together describe the evolution of \ce{^{233}U} in the given system. In the
second row $\Xi$, as seen in equation \ref{eq:Xi_def}, represents the production
of \ce{^{135}Xe} from \ce{^{233}U}. In the second column of the second row are
the processes dependant on the concentration of \ce{^{135}Xe}. $\Upsilon$
represents the proportional rate constant as determined by the multiplication
of $q_{i,v}$ and $r_{v}$ whereas $\Theta$ is given by equation 
\ref{eq:Theta_def}. The third row is blank as the abundance of a continuous type
stream, $c_{n}$, does not change over a burn step. The fourth row is an
addition specific to ADER and not found in the Bateman equations; rather, this
line, and the lines it represents, exists to keep track of the amount of an
isotope that a proportional stream moves simply to provide this information
to the user. The system of matrices seen in figure \ref{fig:burn_matrix} is
solved by SERPENT2 using the CRAM methodology providing updated isotopic
abundances and proportional stream transfer amounts.

    \begin{equation}
    \label{eq:Bateman}
    \begin{split}
        \frac{\mathrm{d}N_{n}(t)}{\mathrm{d}t} = & \sum \limits_{m}^{M} 
        b_{m \rightarrow n} \lambda_{j} N_{j}(t) + \\
        & \sum \limits_{m}^{M}
        \sum \limits_{q}^{Q} a_{m \rightarrow n}^{q}
        \sigma_{q}^{k} \phi(t) N_{k}(t) - \\
        & N_{n}(t) \lambda_{i} - \sum \limits_{d}^{D}
        \sigma_{d}^{n} \phi(t) N_{n}(t) - \\
        & R_{n}(t) N_{n}(t) + F_{n}(t)
    \end{split}
    \end{equation}

    \begin{equation}
    \label{fig:burn_matrix}
        \begin{blockarray}{cccccc}
             &
            \ce{^{233}U} &
            \ce{^{135}Xe} &
            S_{c} &
            S_{p} &
            \mathbb{N} \\
             &
             &
             &
             &
             &
             \\ 
        \begin{block}{c[cccc][c]}
            \ce{^{233}U} &
            -\lambda_{\ce{^{233}U}} + \Gamma &
             &
            f^{S_{c}}_{\ce{^{233}U}} &
             &
            N_{\ce{^{233}U}} \\
            \ce{^{135}Xe} &
            \Xi &
            -\lambda_{\ce{^{135}Xe}} + \Upsilon + \Theta &
             &
             &
            N_{\ce{^{135}Xe}} \\
            S_{c} &
             &
             &
             &
             &
            c_{n}\\
            S_{p} &
             &
             \Upsilon &
             &
             &
             0\\
        \end{block}
        \end{blockarray}
    \end{equation}

\begin{equation}
\label{eq:Gamma_def}
\Gamma = - \sum \limits_{d}^{D} \sigma_{d}^{\ce{^{233}U}} \phi
\end{equation}

\begin{equation}
\label{eq:Xi_def}
\Xi = b_{\ce{^{233}U} \rightarrow \ce{^{135}Xe}} \lambda_{\ce{^{233}U}} + \sum
\limits_{q}^{Q} a_{\ce{^{233}U} \rightarrow \ce{^{135}Xe}} \sigma_{q}^{
\ce{^{233}U}} \phi
\end{equation}

\begin{equation}
\label{eq:Theta_def}
\Theta =-\sum \limits_{d}^{D} \sigma_{d}^{\ce{^{135}Xe}} \phi
\end{equation}

Following the solution of the burnup problem the material compositions are
updated accordingly and the simulation moves on to the next burnup step.

