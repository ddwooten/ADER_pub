A conditions block is the structure through which limits on a SERPENT2 material
are defined. A conditions block consists of, minimally, the elements seen 
below where \texttt{conditions} is the keyword and \texttt{Name} is the unique
identifier given to this specific conditions block.

\begin{lt}
conditions [Name]
\end{lt}

Consider this conditions block \texttt{limits\_block} as defined below.

\begin{li}
conditions limits_block
\end{li}

A conditions block, and all of the limitations it may carry, are applied to
SERPENT2 materials by including the key and value pair, \texttt{cnd [Name]}, in
the material's definition; the information that follows the \texttt{mat}
keyword.

\begin{lt}
... [cnd] [Name] ...
\end{lt}

Building upon the example presented in the introduction, to attach a
conditions block with the name of \texttt{limits\_block} the material definition
line would be amended to look like\ldots

\begin{li}
mat -2.805 FuelSalt vol 1 burn 0 ader cnd limits_block
\end{li}

The same conditions block can be attached to multiple materials with
each material receiving its own unique set of the limitations - this reduces
input duplication.
To add limitations to a conditions block there exist six possibilities: 
ranges, ratios, control tables, oxidation tables, preservation options, and 
optimization options. Each will be covered below in the following subsections.

\subsection{Ranges}\label{ssec:ranges}
Using the \texttt{rng} keyword the fraction of a material,
by percent atom per cubic centimeter ($\frac{\%}{cm^{3}}$) which must adhere
to the recipe of a specific group may be set. This fraction may be either a
single value, seen in method A, or a range of values, as seen in method B.
Values less than 0 are not accepted but there is no upper limit for the 
material fractions. Using a group in a \texttt{rng} structure which is attached
to a material via a condition block is one of two methods to ``attach" a group
to a material - the meaning of this is expanded upon in section 
\ref{ssec:control}

\noindent Method A:

\begin{lt}
rng [grp_name] val [value]
\end{lt}

\noindent Method B:

\begin{lt}
rng [grp_name] [min] [min_value] [max] [max_value]
\end{lt}

Take $g_{u}$ to be the fraction of a material's atomic density which is 
attributed to group $u$, in which case method A sets up the relation seen in
equation \ref{eq:rng_ma} and method B sets up the relation seen in equation
\ref{eq:rng_mb}.


\begin{equation}\label{eq:rng_ma}
g_{u} = \text{\texttt{[value]}}
\end{equation}

\begin{equation}\label{eq:rng_mb}
\text{\texttt{[min\_value]}} \leq g_{u} \leq \text{\texttt{[max\_value]}}
\end{equation}

Now consider the conditions block shown below which was attached to the
material \texttt{FuelSalt} at the beginning of this section, 
\ref{sec:conditions_blocks}.

\begin{li}
conditions limits_block
rng gUF4    min 0.03    max 0.10
\end{li}

By adding a \texttt{rng} input to the conditions block \texttt{limits\_block}
material \texttt{FuelSalt} is now required to have at least 3\% and no more 
than 10\%
of its atomic density accounted for by isotopes which would fit the recipe
of group \texttt{gUF4}. Furthermore, these isotopes may not be counted towards
inclusion in any other group structure attached to the material
\texttt{FuelSalt}.

\subsection{Ratios}\label{ssec:ratios}
The keyword \texttt{rto} is used to describe the permitted relative abundance
of groups within a material. Like \texttt{rng} structures \texttt{rto} entries
can be make according to methods A or B seen below. Values equal to or less than
0 are not accepted. 
Using a group in a \texttt{rng} structure which is attached
to a material via a condition block is one of two methods to ``attach" a group
to a material - the meaning of this is expanded upon in section 
\ref{ssec:control}

\noindent Method A:
\begin{lt}
rto [grp1_name] [val] [value] [grp2] [grp_2_name]
\end{lt}

\noindent Method B:
\begin{lt}
rto [grp1_name] [min] [min_value] [max] [max_value] [grp2] [grp_2_name]
\end{lt}

Method A sets up the relation seen in
equation \ref{eq:rto_ma} and method B sets up the relation seen in equation
\ref{eq:rto_mb}.

\begin{equation}\label{eq:rto_ma}
\frac{g_{1}}{g_{2}} = \text{\texttt{[value]}}
\end{equation}

\begin{equation}\label{eq:rto_mb}
\text{\texttt{[min\_value]}} \leq \frac{g_{1}}{g_{2}} \leq \text{\texttt{[max\_value]}}
\end{equation}

Now, consider adding the following line to the conditions block,
\texttt{limits\_block}, which has served as the example throughout this section.

\begin{li}
conditions limits_block
rng gUF4    min 0.03    max 0.10
rto gFLiBe  min 4.0 max 99  grp2    gUF
\end{li}

This \texttt{rto} line specifies that in the material \texttt{FuelSalt} the
fraction of the material's atomic density which is accounted for by the
group \texttt{gFLiBe} must be no less than 4 times as prevalent and no more
than 99 times as prevalent as the fraction of the material's atomic density
which is accounted for by the group \texttt{gUF} - which is a summation group.

\subsection{Control Tables}\label{ssec:control}
In the ADER framework there is no requirement that all of a material's atomic
density be accounted for by group structures - groups are permitted to
occupy as little or as much of a material's composition as the optimal solution
requires. Consider a group, \texttt{gNi}, which is defined as elemental 
nickel and which is
limited to be less than 2\% of a material's atomic density. One way to meet this
requirement is simply to assign none of the nickel isotopes in the material to
the group \texttt{gNi}. In that case 0\% of the material's atomic density is
accounted for by the group \texttt{gNi}. 

Control tables offer a means to prevent the trivial answer reached above. 
A control table is a list of elements and isotopes. When attached to a material
a control table specifies that any isotopes in the material which are listed
on the control table, or which are a member of an element listed on the control
table, must be fully accounted for by a group structure attached to the 
material. Groups become attached to a material by being a component of an
\texttt{rng} or \texttt{rto} entry in a conditions block which is assigned to
the material in question. In the example above, if there are no other groups
for the nickel isotopes to go into, if nickel is a controlled element in the 
material in question, all of the nickel isotopes in the material would be
forced into the \texttt{gNi} group. An element or isotope not listed on a
control table is referred to as a ``free" element or isotope as these 
constituents are not bound to be in groups.

A control table is defined in the following manner where elements are denoted
by their alphabetic periodic table designation and isotopes by adding a dash
followed by the atomic number of the isotope in question to the alphabetic
periodic table designation for the parent element: i.e. \texttt{U} for uranium
and \texttt{U-233} for \ce{^233U}.

\begin{lt}
control [table_name]
[element_or_isotope]
...
<additional_element_or_isotope>
\end{lt}

For example, consider a control table which specifies that all lithium,
beryllium, \ce{^233U}, and thorium must be accounted for by group structures, as
seen below.

\begin{li}
control c_table
Li Be U-233 Th
\end{li}

A control table is attached to a conditions block via the entry\ldots

\begin{lt}
[cnt] [table_name]
\end{lt}

Attaching the control table, \texttt{c\_table}, from above to the conditions
block used in this section is seen below.

\begin{li}
conditions limits_block
rng gUF4    min 0.03    max 0.10
rto gFLiBe  min 4.0 max 99  grp2    gUF
cnt c_table
\end{li}

\subsection{Oxidation Tables}\label{ssec:oxidation}
Oxidation tables are a means of setting a limitation on a weighted sum which
is done over all the \textit{elements} in a material. Oxidation tables are
constructed as seen below where \texttt{value} is multiplied by
\texttt{weight\_value}, if applicable, which produces the weight in the weighted
sum done over the elements which are enumerated in the oxidation table. Elements
are entered by their alphabetic periodic table designation. Any real number
is a valid input for both \texttt{value} and \texttt{weight\_value}.

\begin{lt}
oxidation [Name]
[first_element] [value] <weight> (weight_value)
<nth_element> (value) <weight> (weight_value)
\end{lt}

As an example consider the oxidation table \texttt{oxi\_table} defined below. 
Any elements excluded from an oxidation table's list are given a default value
of 0.

\begin{li}
oxidation oxi_table
H   1
O   -2
\end{li}

Oxidation tables are attached to conditions blocks either by method A or
method B.

\noindent Method A:

\begin{lt}
oxi [Name] val [target_val]
\end{lt}

\noindent Method B:

\begin{lt}
oxi [Name] [min] [min_val] [max] [max_val]
\end{lt}

Taking $\rho_{e}$ to be the fraction of a material's atomic density which is
attributable to element $e$ an oxidation table, in conjunction with its
implementation in a conditions block attached to said material, establishes 
the relationships seen in equations \ref{eq:oxi_ma} and \ref{eq:oxi_mb} 
corresponding to methods A and B respectively.

\begin{equation}\label{eq:oxi_ma}
\sum_{e}^{E} \rho_{e} v_{e} w_{e} = \text{\texttt{target\_val}}
\end{equation}

\begin{equation}\label{eq:oxi_mb}
\text{\texttt{min\_val}} \leq \sum_{e}^{E} \rho_{e} v_{e} w_{e} \leq 
\text{\texttt{max\_val}}
\end{equation}

Attaching the oxidation table \texttt{oxi\_table} to the conditions block
used throughout this section is demonstrated below using method B.

\begin{li}
conditions limits_block
rng gUF4    min 0.03    max 0.10
rto gFLiBe  min 4.0 max 99  grp2    gUF
cnt c_table
oxi oxi_table min -0.02 max 0.06
\end{li}

\subsection{Preservation}\label{ssec:pres}
Inclusion of the following keyword pair\ldots

\begin{lt}
pres mols
\end{lt}

into a conditions block adds the limitation to the material that any influx
of matter must be equally balanced by a removal of matter with an equal number
of atoms; i.e. the atomic density of a material is not permitted to change due
to mass flows - only nuclear depletion effects. Only mass flows managed by
group-class streams, covered in section \ref{sec:streams}, are included in
this balance. The majority of simulations will
find this option to be necessary; otherwise unintended behavior may result. 

Including this modifier into the conditions block used throughout this section
is seen below.

\begin{li}
conditions limits_block
rng gUF4    min 0.03    max 0.10
rto gFLiBe  min 4.0 max 99  grp2    gUF
cnt c_table
oxi oxi_table min -0.02 max 0.06
pres mols
\end{li}

\subsection{Optimization}\label{ssec:optimization}
As ADER is a linear optimization suite, there must be an optimization target.
Every material cluster, a concept covered in section \ref{sec:streams}, with
at least one group-class stream, must have one and only one optimization
target. Optimization targets are attached to material clusters via a conditions
block which is attached to a member of a material cluster. There are nine
methods to set optimization targets. The direction of optimization, maximization
or minimization, is set in the optimization entry of the conditions table
as well. Many of the optimization targets involve concepts related to streams
which are covered in section \ref{sec:streams}.

\noindent Method A sets the optimization target as the total value of all
feed/reac/redox/remv type streams:

\begin{lt}
opt [dir] [{min; max}] [type] [action] {[feed; reac; redox; remv]} 
\end{lt}

\noindent Method B sets the optimization target as the total value of all 
feed and remv type streams:

\begin{lt}
opt [dir] [{min; max}] [type] [action] [feed_and_remv] 
\end{lt}

\noindent Method C sets the optimization target as the total value of all 
streams: 

\begin{lt}
opt [dir] [{min; max}] [type] [action] [streams] 
\end{lt}

\noindent Method D sets the optimization target as the total value of all 
streams which have a valid SERPENT2 material for both the source and sink:

\begin{lt}
opt [dir] [{min; max}] [type] [action] [transfers]
\end{lt}

\noindent Method E sets the optimization target as the total fraction of a
material's atomic density which is attributed to the named group. This group
\textit{must} have been assigned to a material in the material cluster via
either a \texttt{rng} or \texttt{rto} conditions block entry.

\begin{lt}
opt [dir] [{min; max}] [type] [group] [group_name]
\end{lt}

\noindent Method F sets the optimization target as the total value of all
group-class streams defined using the group of name \texttt{group\_name}.
\begin{lt}
opt [dir] [{min; max}] [type] [spec_stream] [group_name]
\end{lt}

As an example the optimization target of minimizing all feed type streams is
added to the conditions block used throughout this section.

\begin{li}
conditions limits_block
rng gUF4    min 0.03    max 0.10
rto gFLiBe  min 4.0 max 99  grp2    gUF
cnt c_table
oxi oxi_table min -0.02 max 0.06
pres mols
opt dir min type action feed
\end{li}

\subsection{Quick Reference}\label{ssec:conditions_blocks_qr}
\begin{itemize}
\item{Conditions blocks are defined as follows\ldots
\begin{lt}
conditions [Name]
\end{lt}
    }
\item{Conditions blocks are attached to materials using the \texttt{cnd [Name]}
key and value pair in a material's definition - i.e. 
\texttt{mat my\_mat cnd my\_conditions\_block}
    }
\item{Ranges are defined inside a conditions block as follows\ldots
\begin{lt}
rng [grp_name] [{val; min/max}] [value] (max/min) (value)
\end{lt}
    }
\item{Ratios are defined inside a conditions block as follows\ldots
\begin{lt}
rto [grp1_name] [{val; min/max}] [value] (max/min) (value) 
    grp2 [grp2_name]
\end{lt}
    }
\item{Control tables are defined as follows\ldots
\begin{lt}
control [table_name]
[element_or_isotope]
...
<element_or_isotope>
\end{lt}
    }
\item{Control tables are added to conditions blocks as follows\ldots
\begin{lt}
cnt [table_name]
\end{lt}
    }
\item{Oxidation tables are defined as follows\ldots
\begin{lt}
oxidation [table_name]
[element_x] [value] <weight> (weight_value)
[element_n] [value] <weight> (weight_value)
\end{lt}
    }
\item{Oxidation tables are added to conditions blocks as follows\ldots
\begin{lt}
oxi [table_name] [{val; min/max}] [value] (max/min) (value)
\end{lt}
    }
\item{To turn on atom-density conservation for a material, add the following
line to a conditions block attached to said material\ldots
\begin{lt}
pres mols
\end{lt}
    }
\item{Every material cluster with one or more group-class streams must have
one and only one \texttt{opt} entry attached to one of the conditions blocks
which is itself attached to one of the materials in a cluster. \texttt{opt}
entries are added to conditions blocks as follows\ldots
\begin{lt}
opt [dir] [{min; max}] [type] [{action; group; spec_stream}]
    [{feed; feed_and_remv; reac; redox; remv; streams; 
        transfers; group_name}]
\end{lt}
    }
\end{itemize}
