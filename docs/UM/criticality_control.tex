In many nuclear burnup simulations the overall neutron multiplication factor
of the system in question has some desired value or range it should hold. 
ADER provides the user the ability to set overall system $k_{eff}^{analog}$
targets, a minimum and a maximum value, set using the below input. The default
values for \texttt{kmin} and \texttt{kmax} are zero and $\infty$ respectively.

\begin{lt}
kmin [minimum_k_value]
kmax [maximum_k_value]
\end{lt}

For instance, the below input would place a desired lower bound on
$k_{eff}^{analog}$ of 1.0 and an upper bound of 1.001.

\begin{li}
kmin 1.0
kmax 1.001
\end{li}

As discussed in section \ref{sec:theory}, ADER only considers the neutron
multiplication factor of a single SERPENT2 material. This material must be
designated by the user using the keyword and value pair, \texttt{rhow 1.0},
in the material's definition line before the list of its constituent isotopes
such as seen below\ldots

\begin{li}
mat FuelSalt -2.805 vol 1 burn 0 rhow 1.0 ader
\end{li}

Also discussed in section \ref{sec:theory} is the ``probability of absorption"
factor used as an approximation of the effects of the rest of the system outside
of the singular material considered for criticality control. While a detailed
discussion of this approach is left to section \ref{sec:theory} for now it 
will be said
that this method of assessing criticality in systems is expected to produce
reasonable and well-behaved results for systems in which there is a single
material which is both largely responsible for nuclear criticality and weakly
coupled neutronically to its surroundings.

ADER has a limited iteration scheme for criticality search. This scheme is
visually depicted as a flowchart in figure \ref{fig:flowchart}.
At the beginning of
each burnup step ADER builds and solves the optimization problem for all
material clusters. Following this the determined actions for discrete type
streams, both group-class and table-class, are applied to change material
compositions. Following this a new Monte Carlo transport sweep is
executed with all the same population and cycle parameters as the transport sweep
for a burnup step. Following this transport sweep if the system
$k_{eff}^{analog}$ is within the user set bounds, or no iterations remain,
 the simulation proceeds to
the calculation of the nuclear burnup solution. If the $k_{eff}^{analog}$ is
out of bounds \textit{and} iterations remain in the criticality search, ADER
will build and solve the optimization problem for all material clusters, again.
Following this the actions of discrete group-class streams only, discrete
table-class streams are only applied on the very initial iteration of the
criticality search, are again applied to materials to change their compositions.
On every iteration of the criticality search after the initial iteration only
the actions of discrete group-class streams are applied to the material
compositions. The actions of discrete table-class streams are only applied
on the initial iteration of a criticality search. If no criticality search
is being done, if the system neutron multiplication factor has not been
restricted, after the Monte Carlo transport sweep following the initial
application of both discrete group and table-class streams the simulation
calculation proceeds on to the burnup solution. To set the maximum number
of criticality search iterations the input below is used, the default value
of this parameter is 5. The criticality search features of ADER
are only activated by the \texttt{kmax} and \texttt{kmin} keywords - i.e. a 
default value of 5 will not force a simulation through 5 criticality searches
if no criticality search was requested.

\begin{lt}
set ader_trans_iter [value]
\end{lt}

An important consideration of the interaction of streams with multiple
forms, \texttt{cont}, \texttt{disc}, \texttt{prop}, is that if streams of
multiple forms are used which significantly affect system reactivity ADER's
criticality search will only be aware of the effects due to discrete streams.
The effects on criticality from continuous and proportional streams are not
assessed until the burnup calculation has been completed. For
example if ADER has 
determined an amount of \ce{^{233}U} to inject into a material to keep it
critical following a discrete injection of natural Li for chemistry control, and
this uranium is from a continuous stream, the Monte Carlo transport sweep run
before the burnup calculation will only see the negative reactivity effects
of the discrete and instantaneous lithium injection. The positive reactivity
effects of the uranium will be observed in the Monte Carlo transport sweep for
the \textit{next} burnup step.

\subsection{Quick Reference}\label{sec:reactivity_control_qr}
\begin{itemize}
\item{The maximum system $k_{eff}^{analog}$ is set by\ldots
\begin{lt}
kmax [value]
\end{lt}
    }
\item{The minimum system $k_{eff}^{analog}$ is set by\ldots
\begin{lt}
kmin [value]
\end{lt}
    }
\item{The system defaults for \texttt{kmin} and \texttt{kmax} are zero and 
$\infty$ respectively. To use the system $k_{eff}^{analog}$ as a constraint
designate one and only one material with the keyword-value pair 
\texttt{rhow 1.0} in the material's definition and set appropriate
\texttt{kmax} and \texttt{kmin} targets}
\item{The maximum number of criticality search iterations is set by\ldots
\begin{lt}
set ader_trans_iter [value]
\end{lt}
    }
\item{The criticality search feature of ADER is only aware of the reactivity
impacts of discrete form streams, both group and table-class.}
\item{When a criticality search is activated, the reactivity impacts \textit{of
all streams} are assessed as part of the constraints regarding the material
designated by the keyword-value pair \texttt{rhow 1.0}.} 
\end{itemize}
