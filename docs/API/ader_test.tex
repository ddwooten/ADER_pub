ADER includes an extensive suite of unit, integration, and system tests. 
The system tests are not supported by an automated testing environment. Rather,
the system tests, found in the \texttt{System\_Tests} directory of ADER, are
composed of a README file describing the final results of the simulation 
who's input file is provided. To run a system test, execute the given input
file with the specified command line options and compare the simulation output
with what the README file says to expect. For the integration and unit tests
ADER should be compiled with the \texttt{-DADER\_TEST} and 
\texttt{-DADER\_INT\_TEST} options and run with the simulation input found in 
the \texttt{inputs/Test\_Input} directory titled ``\texttt{test\_input.txt}". A
summary report titled ``\texttt{TestResults.test}'' will be produced in the
executing directory. For a detailed accounting of the tests please see the
source code. All tests are either, found in the \texttt{testcases.c} or
have their own file with the name of ``\texttt{TESTADER[NameOfTest].c}". The
file \texttt{serp\_tests.h} is an auxillery header file providing key information
to the test management function, \texttt{runtests.c} which is responsible for
executing the tests found in \texttt{testcases.c}. The standalone test
functions, many of which are integration tests, are called from various places
in the ADER framework, information which can be found in the function's
doc-string. When compiled for unit and integration testing SERPENT2 will not 
work properly for any other purpose and must be recompiled for 
system testing or normal use.
