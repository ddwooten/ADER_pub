Considering that ADER is a source code modification to the SERPENT2 base a basic
understanding of SERPENT2 programming is essential for working in ADER. The
following is \textit{not} intended to be a SERPENT2 API nor will it have near
as much detail. Rather, this section is meant to introduce the reader to key
SERPENT2 structures so that the reader is then prepared to discover more
on their own.
The first aspect of SERPENT2 to grasp is not a function but an array - in fact
it is best of think of this array as an object from the priciples of
object-oriented programming. The \texttt{WDB}, or 
\textbf{W}rite \textbf{D}ata\textbf{B}ase, is a monolithic array of doubles.
\texttt{RDB} is a write-protected cast of \texttt{WDB}. \texttt{WDB} holds
almost all of the program information for SERPENT2 cast as a \texttt{double} and
formatted as an array of linked lists.
The header file, \texttt{locations.h}, holds the initial data structure that
\texttt{WDB} is built from - this is where the initial array location for
the first item in each linked list can be found.
Additionally, and of less importance, are the following arrays: \texttt{PRIVA}
is a doubles array for OpenMP data, \texttt{BUF} is a short term accumulation
array, \texttt{RES1} is a doubles array generally for holding results pulled
from \texttt{BUF}, \texttt{RES2} is an array used for memory optimization in the
burnup routines as is \texttt{RES3} while both are involved with the threaded
behavior of SERPENT2, \texttt{ASCII} is a \texttt{char*} array.
An important concept to keep in mind is that the location of data in all of
these arrays is described by the data in \texttt{WDB}. This data is accessed and
manipulated by functions inside of SERPENT2. While C is not an object-oriented
language by default SERPENT2 behaves as an object-oriented program in that
many of its objects, these data arrays, should only be acted on by specific
methods, functions inside of SEPRENT.  Before diving in to these
functions it should be noted that many variables in SERPENT2 have some name of
``\texttt{ptr}" or a varient thereof. These variables do not indicate C-style
pointers - variables declared with an ``\texttt{*}". Rather they are usually
an array index at which some data of interest can be found. This is mentioned
as the naming convention could cause confusion.

\subsection{AverageTransmuXS}
\textbf{Info}: According to the averaging scheme chosen by the user this function
collects the sub-step averaged transmutation cross sections for the isotopes in 
material mat. These are stored in the PRIVA array. This is thread safe. \\

\noindent \textbf{Inputs}:
\begin{itemize}
\item{long mat}
\item{double t1}
\item{double t2}
\item{long id}
\end{itemize}

\noindent \textbf{Returns}: void


\subsection{BurnMaterials}
\textbf{Info}: Determines which burnup solver will be used for a given mat on a 
given step. \\

\noindent \textbf{Inputs}:
\begin{itemize}
\item{long dep}
\item{long step}
\end{itemize}

\noindent \textbf{Returns}: void


\subsection{BurnMatrixSixe}
\textbf{Info}: Returns the number of non-zero entries in a material's burnup matrix
 without any ADER columns or rows incorporated. \\

\noindent \textbf{Inputs}:
\begin{itemize}
\item{long mat}
\end{itemize}

\noindent \textbf{Returns}: long


\subsection{BurnupCycle}
\textbf{Info}: This is the burnup simulation driver. Schedules transport
caculations, burnup calculations, and data output. \\

\noindent \textbf{Inputs}: None

\noindent \textbf{Returns}: void


\subsection{CalculateTransmuXS}
\textbf{Info}: Calculates current transport sweep transmutation cross sections.
These are used by AverageTransmuXS after having been moved by StoreTransmuXS. 
Thread safe. \\

\noindent \textbf{Inputs}:
\begin{itemize}
\item{long mat}
\item{long id}
\end{itemize}

\noindent \textbf{Returns}: void


\subsection{GetPrivateData}
\textbf{Info}: Retrieves data from PRIVA array, thread safe. \\

\noindent \textbf{Inputs}:
\begin{itemize}
\item{long ptr}
\item{long id}
\end{itemize}

\noindent \textbf{Returns}: double


\subsection{GetText}
\textbf{Info}: Retreives character string from ASCII array. \\

\noindent \textbf{Inputs}:
\begin{itemize}
\item{long ptr}
\end{itemize}

\noindent \textbf{Returns}: char*


\subsection{MaterialBurnup}
\textbf{Info}: Determines the material burnup in MWd/kgHM. \\

\noindent \textbf{Inputs}:
\begin{itemize}
\item{long mat}
\item{double *Nbos}
\item{double *Neos}
\item{double t1}
\item{double t2}
\item{long ss}
\item{long id}
\end{itemize}

\noindent \textbf{Returns}: void


\subsection{MakeBurnMatrix}
\textbf{Info}: Fills a material's burnup matrix with the coefficients from the
Batemann equation. Does not handle ADER materials though it is used in a few ADER
tests. Thread safe. \\

\noindent \textbf{Inputs}:
\begin{itemize}
\item{long mat}
\item{long id}
\end{itemize}

\noindent \textbf{Returns}: struct ccsMatrix*


\subsection{NewItem}
\textbf{Info}: Takes linked-list root, \texttt{root}, in \texttt{WDB} and adds data
block of size \texttt{sz} returning the array index in \texttt{WDB} where the
zeroth index of the new data block is found. NOT THREAD SAFE. \\

\noindent \textbf{Inputs}:
\begin{itemize}
\item{long root}
\item{long sz}
\end{itemize}

\noindent \textbf{Returns}: long


\subsection{NextItem}
\textbf{Info}: Takes an item in a \texttt{WDB} linked-list, \texttt{ptr}, and
returns the next item in that same linked-list. Thread safe. \\

\noindent \textbf{Inputs}:
\begin{itemize}
\item{long ptr}
\end{itemize}

\noindent \textbf{Returns}: long


\subsection{PrepareTransportCycle}
\textbf{Info}: Clears various data buffers, distributes simulation data. Should be
called before any call to TransportCycle(). \\

\noindent \textbf{Inputs}: None

\noindent \textbf{Returns}: void


\subsection{PrintDepOutput}
\textbf{Info}: Produces depletion output. \verb|(_dep.m files)| \\

\noindent \textbf{Inputs}: None

\noindent \textbf{Returns}: void


\subsection{ProcessMaterials}
\textbf{Info}: Handles data initilization for all SERPENT2 materials. \\

\noindent \textbf{Inputs}: None

\noindent \textbf{Returns}: void


\subsection{ReadInput}
\textbf{Info}: Parses user input files - calls data intake routines.\\

\noindent \textbf{Inputs}:
\begin{itemize}
\item{char *inputfile}
\end{itemize}

\noindent \textbf{Returns}: void


\subsection{StoreTransmuXS}
\textbf{Info}: Shuffles material isotopic cross sections into data containers for
the begining of a cycle, the end of a cycle, and previous cycle data. Called
after CalculateTransmuXS but before AverageTransmuXS.\\

\noindent \textbf{Inputs}:
\begin{itemize}
\item{long mat}
\item{long step}
\item{long type}
\item{long id}
\item{long iter}
\end{itemize}

\noindent \textbf{Returns}: void


\subsection{TestParam}
\textbf{Info}: Basic data intake routine checking developer applied limits on
inputs. Called by ReadInput and its subroutines.\\

\noindent \textbf{Inputs}:
\begin{itemize}
\item{char *pname}
\item{char *fname}
\item{long line}
\item{char *val}
\item{long type}
\end{itemize}

\noindent \textbf{Returns}: double


\subsection{TransportCycle}
\textbf{Info}: Primary workhorse of SERPENT2. Runs an entire transport cycle from
inactive cycles to last batch. Should only be called after PrepareTransportCycle.
Not thread safe. \\

\noindent \textbf{Inputs}: None

\noindent \textbf{Returns}: void

