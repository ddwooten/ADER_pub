\documentclass{article}

\begin{document}

\title{ADER: Advanced Depletion Extension for Reprocessing\\
\vspace{10pt}\large{Ver: 1.0 - A SERPENT2 Extension}\\
\vspace{10pt}\large{Application Interface}}

\author{
    Daniel D. Wooten
}

\clearpage
\maketitle
\pagebreak

\tableofcontents
\pagebreak

\section{Preface}\label{sec:pref}
This document is intended for developers of SERPENT2 and specifically those
developers who wish to modify the ADER portions of SERPENT2. For users of
General information about SERPENT2 can be found at the SERPENT2
wiki: \verb|serpent.vtt.ft/mediawiki/index.php/Main_Page|
While ADER specific help can be found in the ADER user manual.
\\
With regards to citing SERPENT2 and ADER, as stated on the SERPENT2 wiki general
reference to SERPENT2 may be provided by - J. Leppanen, M. Pusa, T. Vittanen,
V. Valtavirta, T. Kaltiaisenaho. \textit{``The Serpent Monte Carlo code: Status,
development, and applications in 2013"}. Ann. Nucl. Energy, \textbf{82} (2015)
142 - 150. Reference to ADER may be provided by - D. D. Wooten. \textit{ADER -
Advanced Depletion Extension for Reprocessing}. (2019).
\\
ADER makes extensive use of the open-source library, Clp, part of the COIN-OR
collection of packages. Supporting documentation for Clp can be found at: 
\verb|https://github.com/coin-or/Clp|
Of importance is that Clp is distributed under
the Eclipse Public Licence which is not a copy-left licence but a rather
forgiving open-source licence. Instructions for installing and linking the
Clp libraries can be found in the user's manual.

\section{Introduction}\label{sec:intro}
ADER is a source code extension to SERPENT2. Originally developed by Daniel 
Wooten at the University of California Berkeley ADER provides four key features
to users of SERPENT2 - the ability to define relationships between isotopes,
elements, and chemicals in a SERPENT2 material; the ability to define how
the composition of these SERPENT2 materials may be adjusted; a solution for the
optimal material composition and adjustment schedule; and the incorporation of
these material adjustments into the nuclear burnup solution as determined by
SERPENT2.
\\
In the following sections the inner workings of ADER will be laid out and
explained. The functions will first be organized by themes of collective
action and then elaborated upon alphabetically. A special section, 
\ref{sec:serpent_funs}, will detail a handful of intrinsic SERPENT2 functions,
an understanding of which will facilitate such of ADER.
\\
In section \ref{sec:ader_input} the functions relating to the interpretation
of user input will be detailed. Section \ref{sec:ader_setup} will detail those
functions that sort and link user input into the necessary data structures.
Section \ref{sec:ader_opt} will detail those functions which construct and
solve the optimization problem. Section \ref{sec:ader_burn} will detail those
functions that handle the burnup solution. Section \ref{sec:ader_output} will
detail those functions which output data to the user. Section 
\ref{sec:ader_test} will detail those functions involved with the testing of
ADER. Section \ref{sec:ader_para} will detail some of the methodology
behind parallel computation with ADER.\\
As a final note it needs to be mentioned up front that SERPENT2 makes extensive
use of linked-lists and the term ``\texttt{ptr}". A ``\texttt{ptr}" in this
sense is not a C-style pointer, a variable which contains a memory address, but
the index of a monolithic array at which the desired data may be found. For
instance, if \texttt{NUMBER-OF-CAT-LEGS-PTR = 6} then one would 
expect that at \texttt{CAT-DATA-ARRAY[6]} would be the value ``\texttt{4}".
C-style pointers are referred to simply as pointers.\\

\section{Inside SERPENT2}\label{sec:serpent_funs}
Considering that ADER is a source code modification to the SERPENT2 base a basic
understanding of SERPENT2 programming is essential for working in ADER. The
following is \textit{not} intended to be a SERPENT2 API nor will it have near
as much detail. Rather, this section is meant to introduce the reader to key
SERPENT2 structures so that the reader is then prepared to discover more
on their own.
The first aspect of SERPENT2 to grasp is not a function but an array - in fact
it is best of think of this array as an object from the priciples of
object-oriented programming. The \texttt{WDB}, or 
\textbf{W}rite \textbf{D}ata\textbf{B}ase, is a monolithic array of doubles.
\texttt{RDB} is a write-protected cast of \texttt{WDB}. \texttt{WDB} holds
almost all of the program information for SERPENT2 cast as a \texttt{double} and
formatted as an array of linked lists.
The header file, \texttt{locations.h}, holds the initial data structure that
\texttt{WDB} is built from - this is where the initial array location for
the first item in each linked list can be found.
Additionally, and of less importance, are the following arrays: \texttt{PRIVA}
is a doubles array for OpenMP data, \texttt{BUF} is a short term accumulation
array, \texttt{RES1} is a doubles array generally for holding results pulled
from \texttt{BUF}, \texttt{RES2} is an array used for memory optimization in the
burnup routines as is \texttt{RES3} while both are involved with the threaded
behavior of SERPENT2, \texttt{ASCII} is a \texttt{char*} array.
An important concept to keep in mind is that the location of data in all of
these arrays is described by the data in \texttt{WDB}. This data is accessed and
manipulated by functions inside of SERPENT2. While C is not an object-oriented
language by default SERPENT2 behaves as an object-oriented program in that
many of its objects, these data arrays, should only be acted on by specific
methods, functions inside of SEPRENT.  Before diving in to these
functions it should be noted that many variables in SERPENT2 have some name of
``\texttt{ptr}" or a varient thereof. These variables do not indicate C-style
pointers - variables declared with an ``\texttt{*}". Rather they are usually
an array index at which some data of interest can be found. This is mentioned
as the naming convention could cause confusion.

\subsection{AverageTransmuXS}
\textbf{Info}: According to the averaging scheme chosen by the user this function
collects the sub-step averaged transmutation cross sections for the isotopes in 
material mat. These are stored in the PRIVA array. This is thread safe. \\

\noindent \textbf{Inputs}:
\begin{itemize}
\item{long mat}
\item{double t1}
\item{double t2}
\item{long id}
\end{itemize}

\noindent \textbf{Returns}: void


\subsection{BurnMaterials}
\textbf{Info}: Determines which burnup solver will be used for a given mat on a 
given step. \\

\noindent \textbf{Inputs}:
\begin{itemize}
\item{long dep}
\item{long step}
\end{itemize}

\noindent \textbf{Returns}: void


\subsection{BurnMatrixSixe}
\textbf{Info}: Returns the number of non-zero entries in a material's burnup matrix
 without any ADER columns or rows incorporated. \\

\noindent \textbf{Inputs}:
\begin{itemize}
\item{long mat}
\end{itemize}

\noindent \textbf{Returns}: long


\subsection{BurnupCycle}
\textbf{Info}: This is the burnup simulation driver. Schedules transport
caculations, burnup calculations, and data output. \\

\noindent \textbf{Inputs}: None

\noindent \textbf{Returns}: void


\subsection{CalculateTransmuXS}
\textbf{Info}: Calculates current transport sweep transmutation cross sections.
These are used by AverageTransmuXS after having been moved by StoreTransmuXS. 
Thread safe. \\

\noindent \textbf{Inputs}:
\begin{itemize}
\item{long mat}
\item{long id}
\end{itemize}

\noindent \textbf{Returns}: void


\subsection{GetPrivateData}
\textbf{Info}: Retrieves data from PRIVA array, thread safe. \\

\noindent \textbf{Inputs}:
\begin{itemize}
\item{long ptr}
\item{long id}
\end{itemize}

\noindent \textbf{Returns}: double


\subsection{GetText}
\textbf{Info}: Retreives character string from ASCII array. \\

\noindent \textbf{Inputs}:
\begin{itemize}
\item{long ptr}
\end{itemize}

\noindent \textbf{Returns}: char*


\subsection{MaterialBurnup}
\textbf{Info}: Determines the material burnup in MWd/kgHM. \\

\noindent \textbf{Inputs}:
\begin{itemize}
\item{long mat}
\item{double *Nbos}
\item{double *Neos}
\item{double t1}
\item{double t2}
\item{long ss}
\item{long id}
\end{itemize}

\noindent \textbf{Returns}: void


\subsection{MakeBurnMatrix}
\textbf{Info}: Fills a material's burnup matrix with the coefficients from the
Batemann equation. Does not handle ADER materials though it is used in a few ADER
tests. Thread safe. \\

\noindent \textbf{Inputs}:
\begin{itemize}
\item{long mat}
\item{long id}
\end{itemize}

\noindent \textbf{Returns}: struct ccsMatrix*


\subsection{NewItem}
\textbf{Info}: Takes linked-list root, \texttt{root}, in \texttt{WDB} and adds data
block of size \texttt{sz} returning the array index in \texttt{WDB} where the
zeroth index of the new data block is found. NOT THREAD SAFE. \\

\noindent \textbf{Inputs}:
\begin{itemize}
\item{long root}
\item{long sz}
\end{itemize}

\noindent \textbf{Returns}: long


\subsection{NextItem}
\textbf{Info}: Takes an item in a \texttt{WDB} linked-list, \texttt{ptr}, and
returns the next item in that same linked-list. Thread safe. \\

\noindent \textbf{Inputs}:
\begin{itemize}
\item{long ptr}
\end{itemize}

\noindent \textbf{Returns}: long


\subsection{PrepareTransportCycle}
\textbf{Info}: Clears various data buffers, distributes simulation data. Should be
called before any call to TransportCycle(). \\

\noindent \textbf{Inputs}: None

\noindent \textbf{Returns}: void


\subsection{PrintDepOutput}
\textbf{Info}: Produces depletion output. \verb|(_dep.m files)| \\

\noindent \textbf{Inputs}: None

\noindent \textbf{Returns}: void


\subsection{ProcessMaterials}
\textbf{Info}: Handles data initilization for all SERPENT2 materials. \\

\noindent \textbf{Inputs}: None

\noindent \textbf{Returns}: void


\subsection{ReadInput}
\textbf{Info}: Parses user input files - calls data intake routines.\\

\noindent \textbf{Inputs}:
\begin{itemize}
\item{char *inputfile}
\end{itemize}

\noindent \textbf{Returns}: void


\subsection{StoreTransmuXS}
\textbf{Info}: Shuffles material isotopic cross sections into data containers for
the begining of a cycle, the end of a cycle, and previous cycle data. Called
after CalculateTransmuXS but before AverageTransmuXS.\\

\noindent \textbf{Inputs}:
\begin{itemize}
\item{long mat}
\item{long step}
\item{long type}
\item{long id}
\item{long iter}
\end{itemize}

\noindent \textbf{Returns}: void


\subsection{TestParam}
\textbf{Info}: Basic data intake routine checking developer applied limits on
inputs. Called by ReadInput and its subroutines.\\

\noindent \textbf{Inputs}:
\begin{itemize}
\item{char *pname}
\item{char *fname}
\item{long line}
\item{char *val}
\item{long type}
\end{itemize}

\noindent \textbf{Returns}: double


\subsection{TransportCycle}
\textbf{Info}: Primary workhorse of SERPENT2. Runs an entire transport cycle from
inactive cycles to last batch. Should only be called after PrepareTransportCycle.
Not thread safe. \\

\noindent \textbf{Inputs}: None

\noindent \textbf{Returns}: void



\section{ADER Input} \label{sec:ader_input}
All SEPRENT2 input processing begins with ReadInput. From ReadInput a few of the
following functions call the remaining necessary functions - all to read in and
store user input.\\

\subsection{ADERCreateAderCndEntry}
\textbf{Info}: Calls various functions to read a conditions table. Not a thread
safe. \\

\noindent \textbf{Inputs}:
\begin{itemize}
\item{char* fname}
\item{long line}
\item{char** params}
\item{char* pname}
\item{long np}
\end{itemize}

\noindent \textbf{Returns}: void


\subsection{ADERCreateAderControlEntry}
\textbf{Info}: Calls various functions to read in a control table. Not thread
safe.\\

\noindent \textbf{Inputs}:
\begin{itemize}
\item{char* fname}
\item{long line}
\item{char** params}
\item{char* pname}
\item{long np}
\item{char* word}
\end{itemize}

\noindent \textbf{Returns}: void


\subsection{ADERCreateAderGroupEntry}
\textbf{Info}: Calls functions to read in a group definition. Not thread safe.\\

\noindent \textbf{Inputs}:
\begin{itemize}
\item{char* fname}
\item{long line}
\item{char** params}
\item{char* pname}
\item{long np}
\item{char* word}
\end{itemize}

\noindent \textbf{Returns}: void


\subsection{ADERCreateAderOxidationEntry}
\input{funs/ADERCreateAderOxidationEntry}

\subsection{ADERCreateAderRemovalEntry}
\input{funs/ADERCreateAderRemovalEntry}

\subsection{ADERCreateAderStreamEntry}
\textbf{Info}: Calls various functions to read in an ADER stream entry. Not
thread safe.\\

\noindent \textbf{Inputs}:
\begin{itemize}
\item{char* fname}
\item{long line}
\item{char** params}
\item{char* pname}
\item{long np}
\end{itemize}

\noindent \textbf{Returns}: void


\subsection{ADERReadAderCndCntData}
\input{funs/ADERReadAderCndCntData}

\subsection{ADERReadAderCndData}
\textbf{Info}: Calls various functions to read in the sub-component pieces of
a conditions table. Not thread safe. \\

\noindent \textbf{Inputs}:
\begin{itemize}
\item{char* fname}
\item{long cnd\_ptr}
\item{long j}
\item{long line}
\item{char** params}
\item{char* pname}
\item{long np}
\end{itemize}

\noindent \textbf{Returns}: void


\subsection{ADERReadAderCndOptData}
\textbf{Info}: Reads in opt entry for conditions table. Not thread safe. \\

\noindent \textbf{Inputs}:
\begin{itemize}
\item{char* fname}
\item{long cnd\_ptr}
\item{long j}
\item{long line}
\item{char** params}
\item{char* pname}
\end{itemize}

\noindent \textbf{Returns}: long


\subsection{ADERReadAderCndOxiData}
\input{funs/ADERReadAderCndOxiData}

\subsection{ADERReadAderCndPresData}
\input{funs/ADERReadAderCndPresData}

\subsection{ADERReadAderCndRngData}
\textbf{Info}: Reads in a rng entry for a conditions table. Not thread safe.\\

\noindent \textbf{Inputs}:
\begin{itemize}
\item{char* fname}
\item{long cnd\_ptr}
\item{long j}
\item{long line}
\item{char** params}
\item{char* pname}
\end{itemize}

\noindent \textbf{Returns}: long


\subsection{ADERReadAderCndRtoData}
\input{funs/ADERReadAderCndRtoData}

\subsection{ADERReadAderControlData}
\textbf{Info}: Reads in the elements of a control table. Not thread safe.\\

\noindent \textbf{Inputs}:
\begin{itemize}
\item{char* fname}
\item{long control\_ptr}
\item{long j}
\item{long line}
\item{long np}
\item{char** params}
\item{char* pname}
\end{itemize}

\noindent \textbf{Returns}: void


\subsection{ADERReadAderGroupData}
\textbf{Info}: Reads in a group entry. Not thread safe.\\

\noindent \textbf{Inputs}:
\begin{itemize}
\item{char* fname}
\item{long group\_ptr}
\item{long j}
\item{long line}
\item{long np}
\item{char** params}
\item{char* pname}
\end{itemize}

\noindent \textbf{Returns}: void


\subsection{ADERReadAderGroupIsosData}
\textbf{Info}: Reads in isotopic data for an element in a group entry. Not
thread safe. \\

\noindent \textbf{Inputs}:
\begin{itemize}
\item{long comp\_ptr}
\item{char* fname}
\item{long group\_ptr}
\item{long j}
\item{long line}
\item{char** pname}
\item{char* params}
\item{int num\_isos}
\end{itemize}

\noindent \textbf{Returns}: long


\subsection{ADERReadAderGroupItemData}
\textbf{Info}: Reads in an element entry for a group. Not thread safe.\\

\noindent \textbf{Inputs}:
\begin{itemize}
\item{char* fname}
\item{long group\_ptr}
\item{long j}
\item{long line}
\item{char** params}
\item{char* pname}
\item{long np}
\end{itemize}

\noindent \textbf{Returns}: long


\subsection{ADERReadAderKMaxData}
\textbf{Info}: Reads in the maximum k target. Not thread safe. \\

\noindent \textbf{Inputs}:
\begin{itemize}
\item{char* fname}
\item{long line}
\item{char** params}
\item{char* pname}
\item{long np}
\end{itemize}

\noindent \textbf{Returns}: void


\subsection{ADERReadAderKMinData}
\input{funs/ADERReadAderKMinData}

\subsection{ADERReadAderNegAdens}
\textbf{Info}: Reads the \verb|ader_neg_adens| flag. Not thread safe.\\

\noindent \textbf{Inputs}:
\begin{itemize}
\item{char* fname}
\item{long line}
\item{char** params}
\item{char* pname}
\item{long np}
\end{itemize}

\noindent \textbf{Returns}: void


\subsection{ADERReadAderTransIterData}
\textbf{Info}: Reads the \verb|ader_set_neg_adens| flag. Not thread safe. \\

\noindent \textbf{Inputs}:
\begin{itemize}
\item{char* fname}
\item{long line}
\item{char** params}
\item{char* pname}
\item{long np}
\item{long k}
\end{itemize}

\noindent \textbf{Returns}: long


\subsection{ADERSetMatAderMem}
\textbf{Info}: Initilizes a material's ADER memory block upon the keyword
``ader". Not thread safe. \\

\noindent \textbf{Inputs}:
    \begin{itemize}
        \item{long loc0}
        \item{char** params}
        \item{long np}
        \item{char* pname}
        \item{char* fname}
        \item{long line}
    \end{itemize}

\noindent \textbf{Returns}: void



\section{ADER Setup} \label{sec:ader_setup}
Following the intake of user data, primarily controlled through \texttt{ReadInput},
\texttt{ADERProcessAderMainData} is called by \texttt{main} during the setup
portion of the run. ADER functions with the prefix
 \texttt{ADERProcessMaterialAder}...
generally refer to setup on a per-material basis of linking the system wide
\texttt{ADER} data with the individual materials. 
ADER functions with the prefix
 \texttt{ADERProcessMaterial}...
generally refer to setup on a per-material basis of material-wide \texttt{ADER}
data 
while those functions with the
prefix \texttt{ADERProcessAder}... generally refer to setup on an ADER-wide
basis. \texttt{ADERProcessMaterialAderData} is the primary function which calls
the material specific setup functions, it is called from \texttt{ProcessMaterials}.
It should be noted that throughout the code the terms \texttt{reprocessing} and
\texttt{removal} are used interchangeably in reference to the \texttt{removal}
table structure - as are their short-hands \texttt{remv} and \texttt{repro} 
respectively.
With regards to the data structure in ADER a few terms should be elaborated
upon. The following definitions are to be taken as the primary definition of a
term unless otherwise specified. ``ADER data" refers to the section
of WDB accessed through the \texttt{DATA\_PTR\_ADER|} \texttt{ptr}. 
``Material ADER data" refers to the data accessed for each material through the
\texttt{MATERIAL\_ADER\_DATA} \texttt{ptr}. ``Group isotopes" and ''stream
isotopes" refer to the isotopes which belong to a given group or stream and
which are accessed through the \texttt{ADER\_MAT\_CMP\_ISOS\_PTR|} \texttt{ptr}
and the \texttt{ADER\_MAT\_STREAM\_ISOS\_PTR|} \texttt{ptr} respectively. 
``Material isotopes" refer to the isotope list in a material accessed through
the \texttt{MATERIAL\_PTR\_COMP} \texttt{ptr} while ``material ader
isotopes" refers to the isotope list in a material accessed through the
\texttt{ADER\_MAT\_ISOS\_PTR} \texttt{ptr}. ``Material ADER data" refers
to the ADER related data for a material stored with that material and accessed
through the \texttt{MATERIAL\_ADER\_DATA} \texttt{ptr}. ``Composition
groups" or ``comp groups" refers to groups which have a restricting effect
on a material through either the \texttt{rng} or \texttt{rto} structures. 

\subsection{ADERAddClusterMember}
\textbf{Info}: Adds cluster member to the cluster. 

\noindent \textbf{Inputs}:
\begin{itemize}
\item{long ader\_cluster}
\item{char* ader\_strm\_mem\_id}
\item{double ader\_strm\_mem\_id\_index}
\end{itemize}

\noindent \textbf{Returns}: void


\subsection{ADERCheckMaterialClusterIsotopes}
\textbf{Info}: Loops through a cluster parent's cluster members calling
\texttt{ADERMatchMaterialClusterIsotopes}. \\

\noindent \textbf{Inputs}:
\begin{itemize}
\item{long mat}
\end{itemize}

\noindent \textbf{Returns}: void


\subsection{ADERCheckMaterialRemovalTables}
\textbf{Info}: This function loops through
all the streams of a material looking for removal table type streams. 
If no stream was passed in originally this function calls itself once it has
found a removal table type stream. If
one is found \texttt{ADERCompareMaterialRemovalTables} is called. \\

\noindent \textbf{Inputs}:
\begin{itemize}
\item{long mat}
\item{long passed\_ader\_mat\_stream}
\end{itemize}

\noindent \textbf{Returns}: void


\subsection{ADERCompareMaterialRemovalTables}
\textbf{Info}: Loops through the isotopes of two streams ensuring that none
of them match. This is intended to be used with removal table type streams
for which no isotopes should be shared by removal table type streams in the
same material. \\

\noindent \textbf{Inputs}:
\begin{itemize}
\item{long ader\_mat\_stream}
\item{long mat}
\item{long passed\_ader\_mat\_stream}
\end{itemize}

\noindent \textbf{Returns}: void


\subsection{ADERFindShadowStream}
\textbf{Info}: Searches for a matching stream in a material that is not the
passed in material. This finds shadow streams and returns the negative index
of this stream as destination streams receive the \texttt{ptr} to their source
shadow stream as -1 * index-of-source. Source side streams receive unmodifed
\texttt{ptr}s to their destination shadow streams. 

\noindent \textbf{Inputs}:
\begin{itemize}
\item{long ader\_mat\_stream}
\item{long mat}
\end{itemize}

\noindent \textbf{Returns}: long


\subsection{ADERFindShadowStreamSumStreams}
\textbf{Info}: A recursive function capable of giving all summation streams
in a summation stream, as deeply nested as the user desires, \texttt{ptr}s to
their shadow streams.

\noindent \textbf{Inputs}:
\begin{itemize}
\item{long ader\_mat\_search\_stream}
\item{long ader\_mat\_stream}
\item{long mat}
\end{itemize}

\noindent \textbf{Returns}: void


\subsection{ADERLinkMaterialGroupIsotopes}
\textbf{Info}: Loops through the group-style isotope list that is passed
in and gives these isotopes \texttt{ptr}s to the corresponding
\verb|ADER_MAT_ISO| entry.\\

\noindent \textbf{Inputs}:
\begin{itemize}
\item{long ader\_mat\_ent\_iso}
\item{long mat}
\end{itemize}

\noindent \textbf{Returns}: void


\subsection{ADERLinkMaterialIsotopeIndices}
\textbf{Info}: Goes through a material's composition groups and streams
calling functions to link their isotopes to the material ader isotopes.\\

\noindent \textbf{Inputs}:
\begin{itemize}
\item{long mat}
\end{itemize}

\noindent \textbf{Returns}: void


\subsection{ADERLinkMaterialStreamIsotopes}
\textbf{Info}: Calls \texttt{ADERLinkMaterialGroupIsotopes} for the stream's
isotopes and then loops through any summation groups calling itself on these
summation groups.\\

\noindent \textbf{Inputs}:
\begin{itemize}
\item{long ader\_mat\_strm}
\item{long mat}
\end{itemize}

\noindent \textbf{Returns}: void


\subsection{ADERMatchMaterialClusterIsotopes}
\textbf{Info}: Loops through the isotopes of a cluster member and the cluster
parent ensuring that all cluster members have the same isotopics. \\

\noindent \textbf{Inputs}:
\begin{itemize}
\item{long mat}
\end{itemize}

\noindent \textbf{Returns}: void


\subsection{ADERMergeClusters}
\textbf{Info}: Merges two ADER clusters together - deprecated the half
of the merge which possessed the source side of the stream which led to
the merge. Not thread safe.\\

\noindent \textbf{Inputs}:
\begin{itemize}
\item{long ader\_strm\_dest\_cluster}
\item{long ader\_strm\_src\_cluster}
\end{itemize}

\noindent \textbf{Returns}: void


\subsection{ADERProcessAderClusterMems}
\textbf{Info}: Assigns SERPENT2 materials to an ADER cluster based on ADER
stream information passed from \texttt{ADERProcessAderClusters}. Not thread
safe. \\

\noindent \textbf{Inputs}:
\begin{itemize}
\item{long ader\_strm}
\item{long ader\_strm\_dest\_cluster}
\item{char* ader\_strm\_dest\_id}
\item{long ader\_strm\_src\_cluster}
\item{char* ader\_strm\_src\_id}
\end{itemize}

\noindent \textbf{Returns}: void


\subsection{ADERProcessAderClusters}
\textbf{Info}: Loops through all ADER streams passing off the stream's material
connections to \texttt{ADERProcessAderClusterMems}.\\

\noindent \textbf{Inputs}: None

\noindent \textbf{Returns}: void


\subsection{ADERProcessAderGroupFractions}
\textbf{Info}: Called by \texttt{ADERProcessAderGroups} to normalize user-input
group fractions to 1. Not thread safe. \\

\noindent \textbf{Inputs}:
\begin{itemize}
\item{long grp}
\end{itemize}

\noindent \textbf{Returns}: void


\subsection{ADERProcessAderStreamSourcesAndDests}
\textbf{Info}: Replaces null refrences for stream source and destination with
textual reference to ``NULL". Not thread safe. \\

\noindent \textbf{Inputs}:
\begin{itemize}
\item{long ader\_strm}
\end{itemize}

\noindent \textbf{Returns}: void


\subsection{ADERProcessAderSumGroup}
\textbf{Info}: Give's summation groups \texttt{ptr}s to the groups which
make up their sum. Not thread safe. \\

\noindent \textbf{Inputs}:
\begin{itemize}
\item{long grp}
\end{itemize}

\noindent \textbf{Returns}: void


\subsection{ADERProcessAderGroups}
\textbf{Info}: Calls functions to normalize group fractions to 1 and to link
sumamtion groups to the groups which compose their sum. Loops through all ADER
groups. Not thread safe. \\

\noindent \textbf{Inputs}: None

\noindent \textbf{Returns}: void


\subsection{ADERProcessAderMainData}
\textbf{Info}: Sets default ADER transport loop iteration maximum if the user
has not provided a value. Calls functions to setup data for ADER groups, 
streams, and clusters. Not thread safe. \\

\noindent \textbf{Inputs}: None

\noindent \textbf{Returns}: void


\subsection{ADERProcessAderStreams}
\textbf{Info}: Loops through all ADER streams calling
\texttt{ADERProcessAderStreamSourcesAndDests}. Not thread safe. \\

\noindent \textbf{Inputs}: None

\noindent \textbf{Returns}: void


\subsection{ADERProcessMaterialAderClusterMems}
\textbf{Info}:

\noindent \textbf{Inputs}:
\begin{itemize}
\item{long ader\_cluster}
\item{long mat\_ader\_data}
\end{itemize}

\noindent \textbf{Returns}: void


\subsection{ADERProcessMaterialAderClusterParent}
\textbf{Info}: Gives a material a \texttt{ptr}s to its cluster parent material.

\noindent \textbf{Inputs}:
\begin{itemize}
\item{long ader\_cluster}
\item{long mat\_ader\_data}
\end{itemize}

\noindent \textbf{Returns}: void


\subsection{ADERProcessMaterialAderClusters}
\textbf{Info}: Loops through ADER clusters and calls functions to assign
materials to clusters. Not thread safe.\\

\noindent \textbf{Inputs}:
\begin{itemize}
\item{long mat}
\end{itemize}

\noindent \textbf{Returns}: void


\subsection{ADERProcessMaterialAderCndCntData}
\textbf{Info}: Loops through ADER control tables to give \texttt{mat} a
\texttt{ptr} to the control table its condition table designates. Not thread
safe.\\

\noindent \textbf{Inputs}:
\begin{itemize}
\item{long ader\_cnd\_cnt}
\item{long mat}
\end{itemize}

\noindent \textbf{Returns}: void


\subsection{ADERProcessMaterialAderCndData}
\textbf{Info}: Calls functions to create material group entries and to link
range and ratio restrictions to these groups. Not thread safe. \\

\noindent \textbf{Inputs}:
\begin{itemize}
\item{long ader\_cnd\_ent}
\item{char *ader\_type}
\item{long mat}
\end{itemize}

\noindent \textbf{Returns}: void


\subsection{ADERProcessMaterialAderCndOptData}
\textbf{Info}: Attaches optimization data to materials from ADER condition
tables. Not thread safe. \\

\noindent \textbf{Inputs}:
\begin{itemize}
\item{long ader\_cnd\_opt}
\item{long mat}
\end{itemize}

\noindent \textbf{Returns}: void


\subsection{ADERProcessMaterialAderCndOxiData}
\textbf{Info}: Attaches oxidation table data to a material. Not thread safe.\\

\noindent \textbf{Inputs}:
\begin{itemize}
\item{long ader\_cnd\_oxi}
\item{long mat}
\end{itemize}

\noindent \textbf{Returns}: void


\subsection{ADERProcessMaterialAderCndPresData}
\textbf{Info}: Attaches preservation data from ADER conditions tables to
materials. Not thread safe. \\

\noindent \textbf{Inputs}:
\begin{itemize}
\item{long ader\_cnd\_pres}
\item{long mat}
\end{itemize}

\noindent \textbf{Returns}: void


\subsection{ADERProcessMaterialAderData}
\textbf{Info}: Loops through all SERPENT2 materials checking for those
material's under ADER control. First calls functions to add materials to ADER
clusters, to ensure that all cluster members have matching isotopics, to add
conditions to materials, to connect streams to materials, and to link
various \texttt{ptr}s related to isotopes and their data. Following these
functions shadow streams are processed as all materials must already have
streams to find the shadows ( see the user's manual for this term ). 
The optimization entries are added to materials and then, following a check
on some user data the material cluster composition optimization matrices are
first constructed and allocated space in another function call. Once space has
been allocated these matrices are then filled for the first time. Not thread
safe.\\

\noindent \textbf{Inputs}: None

\noindent \textbf{Returns}: void


\subsection{ADERProcessMaterialAderIsosData}
\textbf{Info}: Creates ADER iso entry for every iso in a material. Not thread
safe. \\

\noindent \textbf{Inputs}:
\begin{itemize}
\item{long mat}
\end{itemize}

\noindent \textbf{Returns}: void


\subsection{ADERProcessMaterialClusterOptEntry}
\textbf{Info}: Copies the optimization target in a cluster to the cluster-parent
material.\\

\noindent \textbf{Inputs}:
\begin{itemize}
\item{long mat}
\end{itemize}

\noindent \textbf{Returns}: void


\subsection{ADERProcessMaterialCndGroupData}
\textbf{Info}: If no group data exists for the given group in the given material
creates an entry in that material for that group and calls functions to process
the group's composition. Not thread safe. \\

\noindent \textbf{Inputs}:
\begin{itemize}
\item{char *ader\_grp\_id}
\item{long mat}
\end{itemize}

\noindent \textbf{Returns}: long


\subsection{ADERProcessMaterialCndRngData}
\textbf{Info}: Copies range data from ADER conditions table to material
condition table.

\noindent \textbf{Inputs}:
\begin{itemize}
\item{long ader\_cnd\_rng}
\item{long ader\_mat\_cmp}
\item{long mat}
\end{itemize}

\noindent \textbf{Returns}: void


\subsection{ADERProcessMaterialCndRtoData}
\textbf{Info}: Copies ratio restrictions from ADER control tables to material
control tables. Not thread safe.\\

\noindent \textbf{Inputs}:
\begin{itemize}
\item{long ader\_cnd\_rto}
\item{long ader\_mat\_cmp}
\item{long mat}
\end{itemize}

\noindent \textbf{Returns}: void


\subsection{ADERProcessMaterialConditions}
\textbf{Info}: Builds a material's conditions block from the designated ADER
conditions block. Calls various function to fill in the pieces. Not thread
safe.\\

\noindent \textbf{Inputs}:
\begin{itemize}
\item{long mat}
\end{itemize}

\noindent \textbf{Returns}: void


\subsection{ADERProcessMaterialGroupComposition}
\textbf{Info}: Copies data from ADER groups to the material groups - whether
those are streams or condition groups. Not thread safe.\\

\noindent \textbf{Inputs}:
\begin{itemize}
\item{long ader\_grp}
\item{long ader\_mat\_ent\_ele\_ptr}
\item{long ader\_mat\_ent\_iso\_ptr}
\item{long ele\_iso\_fix\_check}
\item{long mat}
\item{long stream\_check}
\end{itemize}

\noindent \textbf{Returns}: void


\subsection{ADERProcessMaterialRemovalData}
\textbf{Info}: Sets up removal-table type stream for a material adding the
stream to the material's stream list and calling functions to fill in
the stream data. Not thread safe.\\

\noindent \textbf{Inputs}:
\begin{itemize}
\item{long ader\_mat\_strm}
\item{long ader\_strm}
\item{long mat}
\end{itemize}

\noindent \textbf{Returns}: void


\subsection{ADERProcessMaterialRemovalEle}
\textbf{Info}: Loops through elements in a material's removal table type stream
and creates and fills in these elements' isotopic information. Not thread 
safe. \\

\noindent \textbf{Inputs}:
\begin{itemize}
\item{long ader\_mat\_stream}
\item{long mat}
\end{itemize}

\noindent \textbf{Returns}: void


\subsection{ADERProcessMaterialRemovalEntryData}
\textbf{Info}: Loops through entries of an ADER removal table, being added to 
a material stream, sorting the entries into elemental and isotopic data lists
in the stream as apporpiate. Not thread safe. \\

\noindent \textbf{Inputs}:
\begin{itemize}
\item{long ader\_mat\_strm}
\item{long ader\_rem}
\item{long ele\_iso\_fix\_check}
\item{long mat}
\end{itemize}

\noindent \textbf{Returns}: void


\subsection{ADERProcessMaterialRemovalIsos}
\textbf{Info}: Goes through newly created stream isotopes for a stream based of
a removal table and links these isotopes to the stream element's they should be
linked to. Creates elements if they need creating. Not thread safe. \\

\noindent \textbf{Inputs}:
\begin{itemize}
\item{long ader\_mat\_stream}
\item{long mat}
\end{itemize}

\noindent \textbf{Returns}: void


\subsection{ADERProcessMaterialShadowStreamCompMatrixSection}
\textbf{Info}: Copies a stream's composition matrix index information to the
source side stream.\\

\noindent \textbf{Inputs}:
\begin{itemize}
\item{long ader\_mat\_stream}
\end{itemize}

\noindent \textbf{Returns}: void


\subsection{ADERProcessMaterialShadowStreams}
\textbf{Info}: Loops through a material's streams calling functions to give
these streams \texttt{ptr}s to their shadow streams.  

\noindent \textbf{Inputs}:
\begin{itemize}
\item{long mat}
\end{itemize}

\noindent \textbf{Returns}: void


\subsection{ADERProcessMaterialStreamData}
\textbf{Info}: Copies data from an ADER stream to the material stream. 

\noindent \textbf{Inputs}:
\begin{itemize}
\item{long ader\_mat\_strm}
\item{long ader\_strm}
\item{long mat}
\end{itemize}

\noindent \textbf{Returns}: void


\subsection{ADERProcessMaterialStreamGroupData}
\textbf{Info}: Creates data space for storing a stream's burnup data. Also
calls functions to fill in a stream's elemental and isotopic data. Calls
functions to process summation stream data. Not thread safe.\\

\noindent \textbf{Inputs}:
\begin{itemize}
\item{long ader\_grp}
\item{long ader\_mat\_strm}
\item{long ele\_iso\_fix\_check}
\item{long mat}
\end{itemize}

\noindent \textbf{Returns}: void


\subsection{ADERProcessMaterialStreams}
\textbf{Info}: Loops through all ADER streams assigning them to \texttt{mat} if
they connect to that material. Not thread safe.\\

\noindent \textbf{Inputs}:
\begin{itemize}
\item{long mat}
\end{itemize}

\noindent \textbf{Returns}: void


\subsection{ADERProcessMaterialStreamUnFixedEle}
\textbf{Info}: Loops through a material's isotopes creating a stream element
isotope entry for that isotope if if does not yet have one in its host
stream element. Not thread safe. \\

\noindent \textbf{Inputs}:
\begin{itemize}
\item{long ader\_mat\_stream\_ele}
\item{long ader\_mat\_stream\_iso\_ptr}
\item{long mat}
\end{itemize}

\noindent \textbf{Returns}: void



\section{ADER Optimization} \label{sec:ader_opt}
In the third loop over materials as seen in \texttt{ADERProcessMaterialAderData}
\texttt{ADERCreateMaterialCompMatrix} is called for each material. This function
and its associated calls allocate the space for the material optimization matrix
as well as setting up the associated meta-data - mostly in the form of
\texttt{ptr}s. It should be noted that the material optimization matrix is
incredibly sparse - more than 99\% empty space. An early, and most likely
misguided, design decision led to this matrix being created and stored in full.
For a single SERPENT2 material with all isotopes being tracked about 6 GB of
space is consumed by a single matrix. It is entirely possible, albeit with a
significant architectural overhaul, to forgo storing this matrix and rather
generate it on the fly. This IS NOT how the composition matrix is currently
handled. Rather this information is included as a note on why the memory
footprint of ADER is so high and as a guidepost for future developers.\\
Following the creation of this sparse matrix the final loop over materials in
\texttt{ADERProcessMaterialAderData} calls \texttt{ADERFillMaterialCompMatrix}
which will then fill in the static data for the material optimization matrix.
With regard to many of the \texttt{ADERFill\ldots} functions which take
a SERPENT \texttt{mat} as an input, many of these functions have restrictions
and limitations on what kinds of \texttt{mat}s can be passed to them. Many of
these restrictions are not enforced by code checks - rather they are elaborated
in the comments of these functions. 
An important note is that at various points in the code this matrix, which
represents the linear programming problem of optimizing the material
composition, is referred to as both the ``optimization matrix", the
``composition matrix" and various combinations thereof, including the
abbreviations ``opt matrix" and ``comp matrix" respectively. In this case
``comp" should not be confused with ``cmp" the latter of which is used to
refer to a conditions entry involving a group. \\
Following the initial filling of the static optimization data the program
progresses into the burnup cycles as regulated by \texttt{BurnupCycle} mentioned
in section \ref{sec:serpent_funs}. \texttt{ADERCorrectTransportCycle} is called
after the first transport cycle of each burnup step. This function, and
its associated calls, will fill in the dynamic data for the material
optimization matrices, solve these matrices, incorporate any instantaneous
changes, and re-run the transport cycle checking to ensure the 
neutron multiplication factor, $k_{eff}$, is in the target bounds. If
$k_{eff}$ is, the program proceeds to building and solving the depletion
problem. If not and iterations remain \texttt{ADERCorrectTransportCycle} runs
through the entire process again. The key data gained from all of these steps
just mentioned are the stream values, as determined by ADER, that will
push the material compositions towards their optimum states.
As a final note throughout many functions a variable by the name of \texttt{adj}
can be seen in many places incorporated into if-statements which could
dramatically alter program flow. As of V.1.0 the \texttt{adj} portion of the
algorithms is unused - its value is always set to 0. This variable exists as
part of an early development effort to create algorithmic space for an
interaction scheme inside of the ADER routines. As no such scheme was implemented
\texttt{adj} hangs on as a vestigial organ.

\subsection{ADERAllocateClpMemory}
\textbf{Info}: Allocates system memory for the CLP process. Returns a vector
of pointers to these memory allocations. \\

\noindent \textbf{Inputs}:
\begin{itemize}
\item{long ader\_mat\_matrix\_data}
\end{itemize}

\noindent \textbf{Returns}: double**


\subsection{ADERAverageValue}
\textbf{Info}: Using the SERPENT2 weights for averaging returns an average
of the values passed in over the time interval specified. By using the SERPENT2
weights differint averaging schemes can be employed by SERPENT2 and will be
passed down to this functino as well through the \texttt{WDB} structure.\\

\noindent \textbf{Inputs}:
\begin{itemize}
\item{double bos\_value}
\item{double eos\_value}
\item{double ps1\_value}
\item{double t1}
\item{double t2}
\item{long dep}
\end{itemize}

\noindent \textbf{Returns}: double


\subsection{ADERBuildClpModel}
\textbf{Info}: Fills in the CLP problem data into the arrays pointed to. Returns
the number of non-zero matrix entries in the CLP problem. Not thread safe. \\

\noindent \textbf{Inputs}:
\begin{itemize}
\item{long ader\_mat\_matrix\_data}
\item{double *column\_lower\_bounds}
\item{double *column\_upper\_bounds}
\item{double *index\_column\_starts}
\item{double *objective\_row}
\item{double *row\_lower\_bounds}
\item{double *row\_indices}
\item{double *row\_upper\_bounds}
\item{double *values}
\end{itemize}

\noindent \textbf{Returns}: long


\subsection{ADERBurnMaterials}
\textbf{Info}: Orchestrates the solution of the burnup problem for an ADER
cluster. Distributes this solution and calls various post-burnup checks and
updates, such as \texttt{MaterialBurnup} and \texttt{UpdateCIStop}.\\

\noindent \textbf{Inputs}:
\begin{itemize}
\item{long burn\_ci\_flag}
\item{long mat}
\item{long mode}
\item{long num\_sub\_steps}
\item{long step}
\item{long type}
\end{itemize}

\noindent \textbf{Returns}: void


\subsection{ADERClearAderXSData}
\textbf{Info}: Loops through all materials clearing all material ADER isotope
CUR cross sections. \\

\noindent \textbf{Inputs}: None

\noindent \textbf{Returns}: void


\subsection{ADERClearMaterialCompMatrixClusterMemPresRowBounds}
\textbf{Info}: Sets preservation row bounds in a material's comp matrix to
zero. Not thread safe. \\

\noindent \textbf{Inputs}:
\begin{itemize}
\item{long ader\_mat\_matrix\_data}
\item{long mat}
\end{itemize}

\noindent \textbf{Returns}: void


\subsection{ADERClearPropStreamAmts}
\textbf{Info}:

\noindent \textbf{Inputs}:
\begin{itemize}
\item{long adj}
\item{long mat}
\end{itemize}

\noindent \textbf{Returns}: void


\subsection{ADERClearTargetPropStreamAmts}
\textbf{Info}:

\noindent \textbf{Inputs}:
\begin{itemize}
\item{long ader\_mat\_stream}
\item{long adj}
\end{itemize}

\noindent \textbf{Returns}: void


\subsection{ADERCopyMaterialFlux}
\textbf{Info}: Stores the material flux passed in. Not thread safe. \\

\noindent \textbf{Inputs}:
\begin{itemize}
\item{double flx}
\item{double flx\_new\_avg}
\item{double flx\_old\_avg}
\item{long mat}
\end{itemize}

\noindent \textbf{Returns}: void


\subsection{ADERCorrectTransportCycle}
\textbf{Info}: Called from \texttt{BurnupCycle}. Called after an initial 
transport sweep executed by \texttt{TransportCycle}. Orders the filling-in of
the material composition matrices, their solution, and the incorporation of any
instentaneous changes. Following the incorporation of such changes their
effects on reactivity are assessed with a new transport sweep. After which, if
the multiplication factor of the system is within the user bounds or if the
number of iterations have been exceeded, this function will exit.

\noindent \textbf{Inputs}:
\begin{itemize}
\item{long dep}
\item{long step}
\end{itemize}

\noindent \textbf{Returns}: void


\subsection{ADERCountStream}
\textbf{Info}: Counts the number of streams belonging to the passed in stream -
this number will only be greater than 1 for summation streams. Gives the
stream and all its children their burn matrix index.\\

\noindent \textbf{Inputs}:
\begin{itemize}
\item{long ader\_mat\_stream}
\item{long ader\_mat\_stream\_count}
\item{long num\_rows}
\end{itemize}

\noindent \textbf{Returns}: long


\subsection{ADERCountStreamIsos}
\textbf{Info}: Counts the number of isotopes with non-zero fraction in a stream
and all its children. Returns \texttt{num\_non\_zero\_ents} incremented by this
count.\\

\noindent \textbf{Inputs}:
\begin{itemize}
\item{long ader\_mat\_stream}
\item{long num\_non\_zero\_ents}
\end{itemize}

\noindent \textbf{Returns}: long


\subsection{ADERCreateMaterialClusterMemCompMatrixSection}
\textbf{Info}: Orchestrates the building of the composition matrix for the
portion coming from \texttt{ader\_mat\_cluster\_ent\_mem}. Not thread safe. \\

\noindent \textbf{Inputs}:
\begin{itemize}
\item{long ader\_mat\_cluster\_ent\_mem}
\item{long ader\_mat\_matrix\_data}
\end{itemize}

\noindent \textbf{Returns}: void


\subsection{ADERCreateMaterialCmpGroupCompMatrixSection}
\textbf{Info}: Creates columns in the composition matrix for the passed in
\texttt{cmp} group as well as any rows for any ratios this group may be involved
with. Handles summation rows as well. Not thread safe. \\

\noindent \textbf{Inputs}:
\begin{itemize}
\item{long ader\_mat\_cmp}
\item{long ader\_mat\_matrix\_data}
\end{itemize}

\noindent \textbf{Returns}: void


\subsection{ADERCreateMaterialCompMatrix}
\textbf{Info}: Loops through cluster members of the cluster with parent of
\texttt{mat} calling \texttt{ADERCreateMaterialClusterMemCompMatrixSection} to
create each cluster member's compositional matrix section. Not thread safe. \\

\noindent \textbf{Inputs}:
\begin{itemize}
\item{long mat}
\end{itemize}

\noindent \textbf{Returns}: void


\subsection{ADERCreateMaterialCompMatrixCol}
\textbf{Info}: Adds a column to the material compositional matrix. Not thread
safe. \\

\noindent \textbf{Inputs}:
\begin{itemize}
\item{long ader\_mat\_matrix\_data}
\item{double col\_lower\_bound}
\item{double col\_upper\_bound}
\end{itemize}

\noindent \textbf{Returns}: long


\subsection{ADERCreateMaterialCompMatrixRow}
\textbf{Info}: Creates a new row in a material composition matrix. Not thread
safe. \\

\noindent \textbf{Inputs}:
\begin{itemize}
\item{long ader\_mat\_matrix\_data}
\item{double row\_lower\_bound}
\item{double row\_upper\_bound}
\end{itemize}

\noindent \textbf{Returns}: long


\subsection{ADERCreateMaterialEleCompMatrixSection}
\textbf{Info}: Loops through a material's elements creating the associated
matrix rows and columns as needed and setting initial column and row bounds.
Not thread safe. \\

\noindent \textbf{Inputs}:
\begin{itemize}
\item{long ader\_mat\_matrix\_data}
\item{long mat\_ader\_data}
\end{itemize}

\noindent \textbf{Returns}: void


\subsection{ADERCreateMaterialIsoCompMatrixSection}
\textbf{Info}: Loops through a material's ADER isotopes creating the associated
columns and rows in the material's composition matrix. Also sets initial bounds
for these columns and rows. Not thread safe. \\

\noindent \textbf{Inputs}:
\begin{itemize}
\item{long ader\_mat\_matrix\_data}
\item{long mat\_ader\_data}
\end{itemize}

\noindent \textbf{Returns}: void


\subsection{ADERCreateMaterialOxiCompMatrixSection}
\textbf{Info}: Creates the oxidation row in a material's composition matrix as
well as setting the initial bounds. Not thread safe. \\

\noindent \textbf{Inputs}:
\begin{itemize}
\item{long ader\_mat\_matrix\_data}
\item{long ader\_mat\_oxi}
\end{itemize}

\noindent \textbf{Returns}: void


\subsection{ADERCreateMaterialPresCompMatrixSection}
\textbf{Info}: Creates a row for the preservation entry of a material. Not
thread safe. \\

\noindent \textbf{Inputs}:
\begin{itemize}
\item{long ader\_mat\_matrix\_data}
\item{long ader\_mat\_pres}
\end{itemize}

\noindent \textbf{Returns}: void


\subsection{ADERCreateMaterialRhoCompMatrixSection}
\textbf{Info}: Creates a row for reactivity control in a material's composition
matrix. Not thread safe. \\

\noindent \textbf{Inputs}:
\begin{itemize}
\item{long ader\_mat\_matrix\_data}
\item{long mat\_ader\_data}
\end{itemize}

\noindent \textbf{Returns}: void


\subsection{ADERCreateMaterialStreamCompMatrixSection}
\textbf{Info}: Creates columns and rows for a material's stream's composition
matrix section. Calls itself to deal with summation streams. Not thread safe.
\\

\noindent \textbf{Inputs}:
\begin{itemize}
\item{long ader\_mat\_matrix\_data}
\item{long ader\_mat\_stream}
\end{itemize}

\noindent \textbf{Returns}: void


\subsection{ADERDeallocateTarget}
\textbf{Info}: Frees the memory associated with the passed in array of length
\texttt{target\_size}. NOT THREAD SAFE. \\

\noindent \textbf{Inputs}:
\begin{itemize}
\item{double **target}
\item{long target\_size}
\end{itemize}

\noindent \textbf{Returns}: void


\subsection{ADERFillMaterialClusterMemCompMatrixSection}
\textbf{Info}: Similar to \texttt{ADERCreateMaterialClusterMemCompMatrixSection}
this function calls helper functions to fill in the problem data to a material's
composition matrix. \\

\noindent \textbf{Inputs}:
\begin{itemize}
\item{long ader\_mat\_cluster\_ent\_mem}
\item{long ader\_mat\_matrix\_data}
\end{itemize}

\noindent \textbf{Returns}: void


\subsection{ADERFillMaterialCmpGroupCompMatrixSection}
\textbf{Info}: Calls functions to fill in the group fractions for a comp group's
elemental and isotopic specifications. Calls functions to fill in ratio and
summation information for a group.

\noindent \textbf{Inputs}:
\begin{itemize}
\item{long ader\_mat\_cmp}
\item{long ader\_mat\_matrix\_data}
\item{long mat\_ader\_data}
\end{itemize}

\noindent \textbf{Returns}: void


\subsection{ADERFillMaterialCmpRtoCompMatrixSection}
\textbf{Info}: Fills in composition matrix data to describe a limited ratio
between two comp groups. \\

\noindent \textbf{Inputs}:
\begin{itemize}
\item{long ader\_mat\_cmp\_rto}
\item{long ader\_mat\_matrix\_data}
\item{long ader\_mat\_matrix\_first\_col\_id}
\item{long mat\_ader\_data}
\end{itemize}

\noindent \textbf{Returns}: void


\subsection{ADERFillMaterialCmpSumCompMatrixSection}
\textbf{Info}: Fills in composition matrix data to describe the relationships
between a comp group and its summation groups. \\

\noindent \textbf{Inputs}:
\begin{itemize}
\item{long ader\_mat\_cmp}
\item{long ader\_mat\_matrix\_data}
\item{long mat\_ader\_data}
\end{itemize}

\noindent \textbf{Returns}: void


\subsection{ADERFillMaterialCompMatrix}
\textbf{Info}: Loops through the members of a material cluster calling
\texttt{ADERFillMaterialClusterMemCompMatrixSection} to fill in their 
composition matrix contributions. \\

\noindent \textbf{Inputs}:
\begin{itemize}
\item{long mat}
\end{itemize}

\noindent \textbf{Returns}: void


\subsection{ADERFillMaterialCompMatrixEleData}
\textbf{Info}: Loops through the passed in list of elements,
\texttt{ader\_mat\_ent\_ele}, as taken from a comp group or stream and fills
in the elemental fraction data into the material composition matrix.\\

\noindent \textbf{Inputs}:
\begin{itemize}
\item{long ader\_mat\_ent\_ele}
\item{long ader\_mat\_matrix\_col\_id}
\item{long ader\_mat\_matrix\_data}
\item{long mat\_ader\_data}
\item{double mult}
\item{long sign}
\item{long type}
\end{itemize}

\noindent \textbf{Returns}: void


\subsection{ADERFillMaterialCompMatrixIsoData}
\textbf{Info}: Loops through the passed in isotope list,
\texttt{ader\_mat\_ent\_iso}, from a stream or composition group filling in the
isotopic fraction data into the composition matrix. \\

\noindent \textbf{Inputs}:
\begin{itemize}
\item{long ader\_mat\_ent\_iso}
\item{long ader\_mat\_matrix\_col\_id}
\item{long ader\_mat\_matrix\_data}
\item{long mat\_ader\_data}
\item{double mult}
\item{long sign}
\item{long type}
\end{itemize}

\noindent \textbf{Returns}: void


\subsection{ADERFillMaterialCompMatrixObjRow}
\textbf{Info}: Calls functions to fill in the material composition matrix
objective row given the objective target. \\

\noindent \textbf{Inputs}:
\begin{itemize}
\item{long mat}
\end{itemize}

\noindent \textbf{Returns}: void


\subsection{ADERFillMaterialEleCompMatrixSection}
\textbf{Info}: Loops through a material's elements filling in their data to
the material composition matrix. \\

\noindent \textbf{Inputs}:
\begin{itemize}
\item{long ader\_mat\_matrix\_data}
\item{long mat\_ader\_data}
\end{itemize}

\noindent \textbf{Returns}: void


\subsection{ADERFillMaterialIsoCompMatrixSection}
\textbf{Info}: Loops through a material's ADER isotopes filling in their
composition matrix data. \\

\noindent \textbf{Inputs}:
\begin{itemize}
\item{long ader\_mat\_matrix\_data}
\item{long mat\_ader\_data}
\end{itemize}

\noindent \textbf{Returns}: void


\subsection{ADERFillMaterialObjActFeedAndRemvCompMatrixSection}
\textbf{Info}: Fills in optimization data for feed and removal streams as the
target by looping through all cluster members of cluster parent \texttt{mat}. \\

\noindent \textbf{Inputs}:
\begin{itemize}
\item{long mat}
\end{itemize}

\noindent \textbf{Returns}: void


\subsection{ADERFillMaterialObjActFeedCompMatrixSection}
\textbf{Info}: Fills in optimization data for feed streams as the target looping
through all cluster members of cluster parent \texttt{mat}. \\

\noindent \textbf{Inputs}:
\begin{itemize}
\item{long mat}
\end{itemize}

\noindent \textbf{Returns}: void


\subsection{ADERFillMaterialObjActReacCompMatrixSection}
\textbf{Info}: Fills in optimization data for reac streams as the target by
looping through all cluster members of cluster parent \texttt{mat}.\\

\noindent \textbf{Inputs}:
\begin{itemize}
\item{long mat}
\end{itemize}

\noindent \textbf{Returns}: void


\subsection{ADERFillMaterialObjActRedoxCompMatrixSection}
\textbf{Info}: Fills in optimization data for redox streams as the target by
looping through the cluster members of cluster parent \texttt{mat}. \\

\noindent \textbf{Inputs}:
\begin{itemize}
\item{long mat}
\end{itemize}

\noindent \textbf{Returns}: void


\subsection{ADERFillMaterialObjActRemvCompMatrixSection}
\input{funs/ADERFillMaterialObjActRemvCompMatrixSection}

\subsection{ADERFillMaterialObjActStreamsCompMatrixSection}
\input{funs/ADERFillMaterialObjActStreamsCompMatrixSection}

\subsection{ADERFillMaterialObjActTransfersCompMatrixSection}
\textbf{Info}: Fills in optimization data for streams which transfer substances
between SERPENT2 materials, of which are located in the cluster who's parent is
\texttt{mat} and who's members will be loops through to find such streams. \\

\noindent \textbf{Inputs}:
\begin{itemize}
\item{long mat}
\end{itemize}

\noindent \textbf{Returns}: void


\subsection{ADERFillMaterialObjGrpCompMatrixSection}
\textbf{Info}: Fills in optimization data for target group found in the cluster
members of cluster parent \texttt{mat}. These members are looped through to find
all refernces to the target group.\\

\noindent \textbf{Inputs}:
\begin{itemize}
\item{char* ader\_mat\_opt\_target}
\item{long mat}
\end{itemize}

\noindent \textbf{Returns}: void


\subsection{ADERFillMaterialObjStreamCompMatrixSection}
\textbf{Info}: Fills in optimization data for a specific target stream. All
instances of which will be found by looping through the cluster members of
cluster parent \texttt{mat}.\\

\noindent \textbf{Inputs}:
\begin{itemize}
\item{char* ader\_mat\_opt\_target}
\item{long mat}
\end{itemize}

\noindent \textbf{Returns}: void


\subsection{ADERFillMaterialOxiCompMatrixSection}
\textbf{Info}: Loops through the ADER elements found in \texttt{mat\_ader\_data}
filling in their data to any oxidation rows in the composition matrix.\\

\noindent \textbf{Inputs}:
\begin{itemize}
\item{long ader\_mat\_matrix\_data}
\item{long mat\_ader\_data}
\end{itemize}

\noindent \textbf{Returns}: void


\subsection{ADERFillMaterialPresCompMatrixSection}
\textbf{Info}: This function is more of a placeholder for future developers to
add more \texttt{pres} options. As of V.1.0 there is only one option
and so this function calls \texttt{ADERFillMaterialPresMolsCompMatrixSection}.\\

\noindent \textbf{Inputs}:
\begin{itemize}
\item{long ader\_mat\_cluster\_ent\_mem}
\item{long ader\_mat\_matrix\_data}
\item{long ader\_mat\_pres}
\end{itemize}

\noindent \textbf{Returns}: void


\subsection{ADERFillMaterialPresMolsCompMatrixSection}
\textbf{Info}: Loops through a material's streams incorporating their data into
the preservation row of the composition matrix.\\

\noindent \textbf{Inputs}:
\begin{itemize}
\item{long ader\_mat\_cluster\_ent\_mem}
\item{long ader\_mat\_matrix\_data}
\item{long ader\_mat\_pres}
\end{itemize}

\noindent \textbf{Returns}: void


\subsection{ADERFillMaterialStreamCompMatrixSection}
\textbf{Info}: Fills in composition matrix data for the given stream and any
summation streams it may have via a recurssive call. \\

\noindent \textbf{Inputs}:
\begin{itemize}
\item{long ader\_mat\_cluster\_ent\_mem}
\item{long ader\_mat\_matrix\_data}
\item{long ader\_mat\_stream}
\end{itemize}

\noindent \textbf{Returns}: void


\subsection{ADERGetEigenBias}
\textbf{Info}: Calculates the absorption probability of a reactive material
under ADER control. \\

\noindent \textbf{Inputs}:
\begin{itemize}
\item{long dep}
\item{long mat}
\item{double t1}
\item{double t2}
\end{itemize}

\noindent \textbf{Returns}: double


\subsection{ADERGetIsoBurnMatrixIndex}
\textbf{Info}: Returns the burn matrix index for the specified isotope. Thread
safe.\\

\noindent \textbf{Inputs}:
\begin{itemize}
\item{char* func}
\item{long mat}
\item{long nuc}
\end{itemize}

\noindent \textbf{Returns}: long


\subsection{ADERGetLeakageCorrectionFactor}
\textbf{Info}: Calculates and stores the non-leakage probability. Thread safe.\\

\noindent \textbf{Inputs}:
\begin{itemize}
\item{long dep}
\item{long i}
\item{long step}
\end{itemize}

\noindent \textbf{Returns}: void


\subsection{ADERGetMatEleIsoFrac}
\textbf{Info}: Updates elemental fractions and isotopic elemental fractions
for all elements and material ADER isotopes in a material. \\

\noindent \textbf{Inputs}:
\begin{itemize}
\item{long mat}
\end{itemize}

\noindent \textbf{Returns}: void


\subsection{ADERGetMaterialCompMatrixElement}
\textbf{Info}: A utility function for printing and testing. Returns the value
found at the specified indices in the given composition matrix. Thread safe.\\

\noindent \textbf{Inputs}:
\begin{itemize}
\item{long ader\_mat\_matrix\_data}
\item{long col\_index}
\item{long row\_index}
\end{itemize}

\noindent \textbf{Returns}: double


\subsection{ADERGetMaterialRemovalAmounts}
\textbf{Info}: Loops through a material's streams calling
\texttt{ADERGetStreamRemovalAmounts} on any \texttt{rem} type stream.\\

\noindent \textbf{Inputs}:
\begin{itemize}
\item{long mat}
\item{long i}
\item{double t1}
\item{double t2}
\end{itemize}

\noindent \textbf{Returns}: void


\subsection{ADERGetMaterialShadowStreamIsoFracs}
\textbf{Info}: Provides destination shadow stream isotopes with their proper
fractions as the relative fractions for these isotopes come not from the
material hosting the destination side stream but from the material hosting
the source side stream. \\

\noindent \textbf{Inputs}:
\begin{itemize}
\item{long ader\_mat\_stream}
\end{itemize}

\noindent \textbf{Returns}: void


\subsection{ADERGetMaterialStreamUnFixedEleIsoFracs}
\textbf{Info}: Determines isotopic fractions of elements for a stream with
elements who's isotopic composition is not specified. \\

\noindent \textbf{Inputs}:
\begin{itemize}
\item{long ader\_mat\_matrix\_data}
\item{long ader\_mat\_stream}
\item{long mat}
\end{itemize}

\noindent \textbf{Returns}: void


\subsection{ADERGetStreamRemovalAmounts}
\textbf{Info}: Determines and stores the amount of material moved by a
\texttt{rem} type stream.

\noindent \textbf{Inputs}:
\begin{itemize}
\item{long ader\_mat\_stream}
\item{long mat}
\item{long mat\_ader\_data}
\item{double t1}
\item{double t2}
\end{itemize}

\noindent \textbf{Returns}: void


\subsection{ADERGetTransportInformation}
\textbf{Info}: Calls functions to generate and store needed information
post transport-sweep. \\

\noindent \textbf{Inputs}:
\begin{itemize}
\item{long dep}
\item{long i}
\item{long step}
\end{itemize}

\noindent \textbf{Returns}: void


\subsection{ADERMoveBosEosPs1Values}
\textbf{Info}: Shuffles values amoung the specified indices to match SERPENT2
averaging.

\noindent \textbf{Inputs}:
\begin{itemize}
\item{long avg\_index}
\item{long cur\_index}
\item{long bos\_index}
\item{long eos\_index}
\item{long ps1\_index}
\item{long mat}
\item{long step}
\item{long iter}
\end{itemize}

\noindent \textbf{Returns}: void


\subsection{ADERMoveCrossSection}
\textbf{Info}: Loops through a material's ADER isotopes calling
\texttt{ADERMoveBosEosPs1Values} for an isotopes various cross sections.\\

\noindent \textbf{Inputs}:
\begin{itemize}
\item{long mat}
\item{long step}
\item{long iter}
\end{itemize}

\noindent \textbf{Returns}: void


\subsection{ADERNormalizeCrossSection}
\textbf{Info}: Loops through a material's ADER isotopes normalizing their 
cross sections by \texttt{flx}. \\

\noindent \textbf{Inputs}:
\begin{itemize}
\item{double flx}
\item{long mat}
\end{itemize}

\noindent \textbf{Returns}: void


\subsection{ADEROperateMaterial}
\textbf{Info}: If any one function could be considered the workhorse of ADER, 
it would be this one. This function, called from
\texttt{ADERCorrectTransportCycle} orders and executes the updating of a
cluster's composition matrix as well as the determination of the optimal
solution. Numerous components of the composition matrix require updating with
each ADER iteration. The majority of these updates are due to changing isotopic
compositions and possible changes to material densities.
The \texttt{stream\_counter} variable is necessary as ADER simulations with
only prescriptive streams will have no ``optimal" solution and as such the
solving of such a problem should be skipped. Once the various updates to the
composition matrix have been completed the solver function, 
\texttt{ADEROperateMaterialCompMatrix} is called.\\

\noindent \textbf{Inputs}:
\begin{itemize}
\item{long adj}
\item{long dep}
\item{long i}
\item{long mat}
\item{long step}
\item{double t1}
\item{double t2}
\end{itemize}

\noindent \textbf{Returns}: void


\subsection{ADEROperateMaterialCompMatrix}
\textbf{Info}: Manages the construction, solution, and distribution of solution
data for the optimization problem. \\

\noindent \textbf{Inputs}:
\begin{itemize}
\item{long adj}
\item{long i}
\item{long mat}
\item{long step}
\end{itemize}

\noindent \textbf{Returns}: void


\subsection{ADERParseClpSolution}
\textbf{Info}: Manages the distribution of the optimization solution to the
various comp groups and streams. \\

\noindent \textbf{Inputs}:
\begin{itemize}
\item{long adj}
\item{long i}
\item{long mat}
\item{long step}
\item{double *solution}
\end{itemize}

\noindent \textbf{Returns}: void


\subsection{ADERParseStreamClpSolution}
\textbf{Info}: Recurssive function for distributing optimization solution data
to streams, which may themselves have summation streams. \\

\noindent \textbf{Inputs}:
\begin{itemize}
\item{long ader\_mat\_stream}
\item{long adj}
\item{double *solution}
\end{itemize}

\noindent \textbf{Returns}: void


\subsection{ADERProcessMaterialDiscStreamEffects}
\textbf{Info}: Updates material isotopics and density for discrete stream
transfers. \\

\noindent \textbf{Inputs}:
\begin{itemize}
\item{long ader\_mat\_stream}
\item{long adj}
\item{long i}
\item{long mat}
\end{itemize}

\noindent \textbf{Returns}: void


\subsection{ADERProcessMaterialShadowStreamEleAndIsoFracs}
\textbf{Info}: Manages updating destination side shadow streams with
elemental and isotopic fractions from the source side streams.\\

\noindent \textbf{Inputs}:
\begin{itemize}
\item{long mat}
\end{itemize}

\noindent \textbf{Returns}: void


\subsection{ADERScoreCrossSection}
\textbf{Info}: Sorts cross section information obtained from
\texttt{CalculateTransmuXS} into the appropriate ADER containers. \\

\noindent \textbf{Inputs}:
    \begin{itemize}
        \item{long abs}
        \item{long E}
        \item{long id}
        \item{long mat}
        \item{long nuc}
        \item{long rea}
        \item{double value}
    \end{itemize} 

\noindent \textbf{Returns}: void


\subsection{ADERSetMaterialCompMatrixClusterMemColBounds}
\textbf{Info}: Manages the collection of column bounds information and the
assignment of these column bounds for a particular cluster member - primarily
 by calling other functions.\\

\noindent \textbf{Inputs}:
\begin{itemize}
\item{long mat}
\end{itemize}

\noindent \textbf{Returns}: void


\subsection{ADERSetMaterialCompMatrixClusterMemPresRowBounds}
\textbf{Info}: Adjusts preservation row bounds.\\

\noindent \textbf{Inputs}:
\begin{itemize}
\item{long ader\_mat\_matrix\_data}
\item{long mat}
\item{double value}
\end{itemize}

\noindent \textbf{Returns}: void


\subsection{ADERSetMaterialCompMatrixClusterMemRemovalTableRowBounds}
\textbf{Info}: Adjusts elemental and isotopic balance row bounds to account
for the removal effects of \texttt{ader\_mat\_stream}. Also calls
\texttt{ADERSetMaterialCompMatrixClusterMemPresRowBounds}.\\

\noindent \textbf{Inputs}:
\begin{itemize}
\item{long ader\_mat\_matrix\_data}
\item{long ader\_mat\_stream}
\item{long dep}
\item{long mat}
\item{double t1}
\item{double t2}
\end{itemize}

\noindent \textbf{Returns}: void


\subsection{ADERSetMaterialCompMatrixClusterMemRhoRowEntries}
\textbf{Info}: Loops through a material's ader isotopes, if this material is
under reactivity control, filling in the composition matrix reactivity data
for each isotope. \\

\noindent \textbf{Inputs}:
\begin{itemize}
\item{long dep}
\item{long mat}
\item{double t1}
\item{double t2}
\end{itemize}

\noindent \textbf{Returns}: void


\subsection{ADERSetMaterialCompMatrixClusterMemRowBounds}
\textbf{Info}: Sets elemental and isotopic balance row bounds for \texttt{mat}.
Loops through the streams of the same calling 
\texttt{ADERSetMaterialCompMatrixClusterMemRemovalTableRowBounds} for rem type
streams.\\

\noindent \textbf{Inputs}:
\begin{itemize}
\item{long dep}
\item{long i}
\item{long mat}
\item{double t1}
\item{double t2}
\end{itemize}

\noindent \textbf{Returns}: void


\subsection{ADERSetMaterialCompMatrixColBounds}
\textbf{Info}: A utility function for adjusting a column bound in a composition
matrix.\\

\noindent \textbf{Inputs}:
\begin{itemize}
\item{long bound}
\item{long increment}
\item{long mat\_matrix\_col\_id}
\item{long mat\_matrix\_data}
\item{double value}
\end{itemize}

\noindent \textbf{Returns}: void


\subsection{ADERSetMaterialCompMatrixElement}
\textbf{Info}: A utility function for setting the value of a composition matrix
at a given index.\\

\noindent \textbf{Inputs}:
\begin{itemize}
\item{long index\_col}
\item{long index\_row}
\item{long ader\_mat\_matrix\_data}
\item{double value}
\end{itemize}

\noindent \textbf{Returns}: void


\subsection{ADERSetMaterialCompMatrixRowBounds}
\textbf{Info}: A utility function for setting the row bounds for a row in a
composition matrix.\\

\noindent \textbf{Inputs}:
\begin{itemize}
\item{long bound}
\item{long increment}
\item{long mat\_matrix\_data}
\item{long mat\_matrix\_row\_id}
\item{double value}
\end{itemize}

\noindent \textbf{Returns}: void


\subsection{ADERSetShadowStreamRemovalAmount}
\textbf{Info}: Passes removal amounts from rem type stream elements and isotopes
to their destination side shadow stream equivalents.\\

\noindent \textbf{Inputs}:
\begin{itemize}
\item{long ader\_mat\_stream}
\item{long ele\_id}
\item{long iso\_id}
\item{double value}
\end{itemize}

\noindent \textbf{Returns}: void


\subsection{ADERSolveClpModel}
\textbf{Info}: Converts the SERPENT2 representation of the optimization problem
into the form exepcted by the CLP library. Passes this constructed problem
to the CLP library from which this function retreives the problem solution. \\

\noindent \textbf{Inputs}:
\begin{itemize}
\item{double *column\_lower\_bounds}
\item{double *column\_upper\_bounds}
\item{double *index\_column\_starts}
\item{long num\_cols}
\item{long num\_ent}
\item{long num\_rows}
\item{double *objective\_row}
\item{long opt\_dir}
\item{double *row\_lower\_bounds}
\item{double *row\_indices}
\item{double *row\_upper\_bounds}
\item{double *solution}
\item{double *values}
\item{long mat}
\end{itemize}

\noindent \textbf{Returns}: void


\subsection{ADERUpdateMaterialDiscStreamEffects}
\textbf{Info}: Loops through a material's streams passing discrete type streams
off to \texttt{ADERProcessMaterialDiscStreamEffects} so that these stream's
changes to the material may be promptly incorporated.\\

\noindent \textbf{Inputs}:
\begin{itemize}
\item{long adj}
\item{long i}
\item{long mat}
\end{itemize}

\noindent \textbf{Returns}: void



\section{ADER Burnup} \label{sec:ader_burn}
Following the solution of the material optimization problem as handled by
\texttt{ADERCorrectTransportCycle}, \texttt{BurnupCycle} will shortly call
\texttt{BurnMaterials} which will manage the burnup solution for the current 
step. \texttt{BurnMaterials} will then pass off ADER cluster parent materials
to \texttt{ADERBurnMaterials} which manages the burnup solution for ADER
clusters. Much like the composition matrices, material's which are linked
by ADER streams must have their burnup matrices solved as a whole. The burnup
matrix as produced by ADER is ill-behaved when solved with TTA methods. As such,
only the CRAM solution method is used on ADER material burnup problems. The
burnup matrix is built one column at a time with no values dropped. It is stored
as in a dense column-major format. 

\subsection{ADERBurnMaterials}
\textbf{Info}: Orchestrates the solution of the burnup problem for an ADER
cluster. Distributes this solution and calls various post-burnup checks and
updates, such as \texttt{MaterialBurnup} and \texttt{UpdateCIStop}.\\

\noindent \textbf{Inputs}:
\begin{itemize}
\item{long burn\_ci\_flag}
\item{long mat}
\item{long mode}
\item{long num\_sub\_steps}
\item{long step}
\item{long type}
\end{itemize}

\noindent \textbf{Returns}: void


\subsection{ADERGetBurnMatrixSizeData}
\textbf{Info}: Returns an array of pointers to memory allocations for the
burnup problem.

\noindent \textbf{Inputs}:
\begin{itemize}
\item{long mat}
\end{itemize}

\noindent \textbf{Returns}: double**


\subsection{ADERMakeBurnMatrix}
\textbf{Info}: This function replicates, through itself and its child-functions,
the functionality of \texttt{MakeBurnMatrix} with the addition of functionality
as needed by ADER. The burnup matrix is stored in a dense column-major format. 
Cross sections are pulled from \texttt{rea} structures which are attached to 
\texttt{nuc}s. As such, the sequence of functions \texttt{CalculateTransmuXS}
\texttt{StoreTransmuXS} must be called for each cluster member such that the
cross sections for that cluster member are used when filing in the burnup matrix
data as the two mentioned functions move the XS information from the material
\texttt{dep} structures to the \texttt{rea} structures.

\noindent \textbf{Inputs}:
\begin{itemize}
\item{struct ccsMatrix burn\_matrix}
\item{double* col\_vector}
\item{long mat}
\item{long num\_ents}
\item{long num\_rows}
\item{long step}
\item{long step\_type}
\item{double t1}
\item{double t2}
\end{itemize}

\noindent \textbf{Returns}: void


\subsection{ADERMapDensityVector}
\textbf{Info}: If used in ``receive" mode this function pulls isotopic abundance
and proportional stream transfer amounts from the burnup problem solution
vector. If used in ``send" mode this function fills a vector with isotopic
abundance information and continuous stream injection data. Thread safe. \\

\noindent \textbf{Inputs}:
\begin{itemize}
\item{double* ader\_burn\_matrix\_N}
\item{double* ader\_burn\_matrix\_starts}
\item{long adj}
\item{long direction}
\item{long mat}
\item{long predictor}
\item{double total\_time}
\end{itemize}

\noindent \textbf{Returns}: void


\subsection{ADERMapDensityVectorStream}
\textbf{Info}: Called by \texttt{ADERMapDensityVector} to tunnel into streams
and their possible summation streams and thus gather, fill, and send the
requsite information.\\

\noindent \textbf{Inputs}:
\begin{itemize}
\item{double* ader\_burn\_matrix\_N}
\item{long ader\_mat\_stream}
\item{long adj}
\item{long direction}
\item{double total\_time}
\end{itemize}

\noindent \textbf{Returns}: void


\subsection{ADERProcessBurnMatrixContStream}
\textbf{Info}: Fills in and stores the column in the burnup matrix corresponding
to \texttt{ader\_mat\_stream} and any summation streams it may have.\\

\noindent \textbf{Inputs}:
\begin{itemize}
\item{struct ccsMatrix* burn\_matrix}
\item{long ader\_mat\_stream}
\item{double* col\_vector}
\item{long entry\_number}
\item{long mat}
\item{long num\_rows}
\item{long* return\_array}
\end{itemize}

\noindent \textbf{Returns}: long


\subsection{ADERProcessBurnMatrixFissionYield}
\textbf{Info}: Manages fission data and fission yield data for an isotope in
the burnup matrix.\\

\noindent \textbf{Inputs}:
\begin{itemize}
\item{long ader\_mat\_iso}
\item{double* col\_vector}
\item{long fission\_yield\_data}
\item{long mat}
\item{double mat\_flux}
\item{long nuc}
\item{long omp\_id}
\item{long rea}
\item{long type}
\end{itemize}

\noindent \textbf{Returns}: void


\subsection{ADERProcessBurnMatrixPropStream}
\textbf{Info}: Determines and fills proportional stream contributions to the
burnup matrix. \\

\noindent \textbf{Inputs}:
\begin{itemize}
\item{long ader\_mat\_ader\_iso}
\item{long ader\_mat\_stream}
\item{double* col\_vector}
\item{long mat}
\end{itemize}

\noindent \textbf{Returns}: void


\subsection{ADERProcessBurnMatrixTransmutationAndDecay}
\textbf{Info}: Manages contributions from the transmutation and decay of
isotopes and their children to the burnup matrix.\\

\noindent \textbf{Inputs}:
\begin{itemize}
\item{long ader\_mat\_iso}
\item{double* col\_vector}
\item{long mat}
\item{double mat\_flux}
\item{long nuc}
\item{long omp\_id}
\item{long rea}
\item{long reaction\_product\_nuc}
\item{long type}
\end{itemize}

\noindent \textbf{Returns}: void


\subsection{ADERStoreBurnMatrixColumn}
\textbf{Info}: Stores a column of the burnup matrix into the ccsMatrix
structure.

\noindent \textbf{Inputs}:
\begin{itemize}
\item{struct ccsMatrix* burn\_matrix}
\item{long col\_index}
\item{double* col\_vector}
\item{long entry\_number}
\item{long num\_rows}
\end{itemize}

\noindent \textbf{Returns}: long



\section{ADER Output} \label{sec:ader_output}
Following the solution of the burnup problem program flow will return to
\texttt{BurnupCycle}. Following each depletion step \texttt{BurnupCycle} will
call \texttt{PrintDepOutput} to reproduce ALL of SERPENT2's output each
burnup step. In a loop over materials \texttt{PrintDepOutput} will call
\texttt{ADERPrintOutput}. Also included in this section are several ADER
utility functions that have no impact or use for the casual observer - rather
many of these functions can be activated via compilation flags to output
copious amounts of program data that could be useful for debugging or
optimizing.

\subsection{ADERGetBurnMatrixValue}
\textbf{Info}: A utility function for testing and printing. Returns the value
of the burnup matrix at the given indices. Thread safe.\\

\noindent \textbf{Inputs}:
\begin{itemize}
\item{long col\_index}
\item{struct ccsMatrix* burn\_matix}
\item{long row\_index}
\end{itemize}

\noindent \textbf{Returns}: double


\subsection{ADERGetTargetRemovalAmount}
\textbf{Info}: Determines the net amount of an isotope or element that is moved
by all of a material's \texttt{rem} type streams by looping through all of a
material's streams and calling \texttt{ADERGetStreamTargetRemovalAmount}.
Thread safe. \\

\noindent \textbf{Inputs}:
\begin{itemize}
\item{long mat}
\item{long ele}
\item{long iso}
\end{itemize}

\noindent \textbf{Returns}: double


\subsection{ADERGetStreamTargetRemovalAmount}
\textbf{Info}: Returns the amount of a target element or isotope which is moved
by \texttt{ader\_mat\_stream}. Thread safe.\\

\noindent \textbf{Inputs}:
\begin{itemize}
\item{long ader\_mat\_stream}
\item{long ele}
\item{long iso}
\end{itemize}

\noindent \textbf{Returns}: double


\subsection{ADEROutputBurnMatrixAsCsv}
\textbf{Info}: A utility function activated by the compilation flag
-DADER\_DIAG. Outputs the specified burnup matrix as an ASCII compliant csv
file. Thread safe. \\

\noindent \textbf{Inputs}:
\begin{itemize}
\item{struct ccsMatrix* burn\_matrix}
\item{long ader\_mat\_burn\_matrix\_num\_rows}
\item{long mat}
\item{long step}
\item{long sub\_step}
\end{itemize}

\noindent \textbf{Returns}: void


\subsection{ADEROutputMaterialCompMatrixAsCsv}
\textbf{Info}: Utility function activated by the compliation flag DADER\_DIAG.
Outputs the given composition matrix as a csv file. Thread safe.\\

\noindent \textbf{Inputs}:
\begin{itemize}
\item{long ader\_mat\_matrix\_data}
\item{long cluster\_num}
\end{itemize}

\noindent \textbf{Returns}: void


\subsection{ADEROutputMaterialCompMatrixData}
\textbf{Info}: A utility function activated with the compilation flag 
-DADER\_DIAG. Outputs a json formatted file with a complete description of the
ADER environment.\\

\noindent \textbf{Inputs}: None

\noindent \textbf{Returns}: void


\subsection{ADEROutputMaterialCompMatrixStreamData}
\textbf{Info}: A utility function to tunnel into streams and their summation
streams for \texttt{ADEROutputMaterialCompMatrixData} and put their data
into the json file.\\

\noindent \textbf{Inputs}:
\begin{itemize}
\item{long ader\_mat\_cluster\_mem}
\item{long ader\_mat\_stream}
\item{long ader\_mat\_stream\_sum\_stream\_check}
\item{FILE* fp}
\item{long level}
\item{long tab\_level}
\item{long tab\_length}
\end{itemize}

\noindent \textbf{Returns}: long


\subsection{ADERPrintCrossSections}
\textbf{Info}: A utility function which outputs material ader isotope cross 
sections to file.\\ 

\noindent \textbf{Inputs}:
\begin{itemize}
\item{long dep}
\item{long i}
\item{long mat}
\item{long step}
\item{double t1}
\item{double t2}
\end{itemize}

\noindent \textbf{Returns}: void


\subsection{ADERPrintFinalStepCrossSections}
\textbf{Info}: Utliity function which outputs material ader isotope cross
sections AFTER the conclusion of ADER iterations. This is such that the ADER
cross sections have been updated with the latest material flux whereas
in \texttt{ADERPrintCrossSections} the ADER cross sections reported are those
the isotopes had before the most recent discrete streme adjustment. \\ 

\noindent \textbf{Inputs}:
\begin{itemize}
\item{long dep}
\item{long mat}
\item{long step}
\item{double t1}
\item{double t2}
\end{itemize}

\noindent \textbf{Returns}: void


\subsection{ADERPrintIndentedOutput}
\textbf{Info}: Utility function to ease tab-setting in json files. \\

\noindent \textbf{Inputs}:
\begin{itemize}
\item{FILE* fp}
\item{char* print\_data}
\item{long tab\_length}
\item{long tab\_level}
\end{itemize}

\noindent \textbf{Returns}: void


\subsection{ADERPrintListsHierarchy}
\textbf{Info}: A utliity function activated by -DADER\_DIAG which outputs to
a text file a summary of ADER \texttt{WDB} address information.\\

\noindent \textbf{Inputs}: None

\noindent \textbf{Returns}: void


\subsection{ADERPrintMaterialStreamIsotopes}
\textbf{Info}: Utility function activated by -DADER\_DIAG. Called by
\texttt{ADERPrintListsHierarchy} to tunnel into stream and summation stream
data.\\

\noindent \textbf{Inputs}:
\begin{itemize}
\item{long ader\_mat\_stream}
\item{FILE *fp}
\item{long mat}
\end{itemize}

\noindent \textbf{Returns}: void


\subsection{ADERPrintOutput}
\textbf{Info}: NOT A UTILITY. Prints to the ``\_dep.m" output file ADER
problem information associated with \texttt{burn\_mat}. Calls
\texttt{ADERPrintOutputStreamData} to tunnel into streams and any summation
streams and output their data.\\

\noindent \textbf{Inputs}:
\begin{itemize}
\item{long burn\_mat}
\item{FILE* fp}
\item{char* mat\_name}
\end{itemize}

\noindent \textbf{Returns}: void


\subsection{ADERPrintOutputStreamData}
\textbf{Info}: Called by \texttt{ADERPrintOutput} to facilitate printing of
ADER stream data into the ``\_dep.m" file.\\

\noindent \textbf{Inputs}:
\begin{itemize}
\item{long ader\_mat\_stream}
\item{FILE* fp}
\item{char* mat\_name}
\end{itemize}

\noindent \textbf{Returns}: void


\subsection{ADERPrintSumStreams}
\textbf{Info}: Utility function called by \texttt{ADERPrintListsHierarchy}
to output summation stream data.\\

\noindent \textbf{Inputs}:
\begin{itemize}
\item{long ader\_mat\_stream\_sum\_ent}
\item{FILE* fp}
\item{int sum\_level}
\end{itemize}

\noindent \textbf{Returns}: void



\section{ADER Testing} \label{sec:ader_test}
ADER includes an extensive suite of unit, integration, and system tests. 
The system tests are not supported by an automated testing environment. Rather,
the system tests, found in the \texttt{System\_Tests} directory of ADER, are
composed of a README file describing the final results of the simulation 
who's input file is provided. To run a system test, execute the given input
file with the specified command line options and compare the simulation output
with what the README file says to expect. For the integration and unit tests
ADER should be compiled with the \texttt{-DADER\_TEST} and 
\texttt{-DADER\_INT\_TEST} options and run with the simulation input found in 
the \texttt{inputs/Test\_Input} directory titled ``\texttt{test\_input.txt}". A
summary report titled ``\texttt{TestResults.test}'' will be produced in the
executing directory. For a detailed accounting of the tests please see the
source code. All tests are either, found in the \texttt{testcases.c} or
have their own file with the name of ``\texttt{TESTADER[NameOfTest].c}". The
file \texttt{serp\_tests.h} is an auxillery header file providing key information
to the test management function, \texttt{runtests.c} which is responsible for
executing the tests found in \texttt{testcases.c}. The standalone test
functions, many of which are integration tests, are called from various places
in the ADER framework, information which can be found in the function's
doc-string. When compiled for unit and integration testing SERPENT2 will not 
work properly for any other purpose and must be recompiled for 
system testing or normal use.


\section{ADER in Parallel} \label{sec:ader_para}
With regards to parallel computation there are two key facts: ADER is able to
take advantage of shared-memory parallelization through the OpenMP interface,
and ADER is unable to make use of distributed-memory parallelization which for
SERPENT2 is hanlded through the MPI interface. There is no inherent reason ADER
can not function inside of a distributed-memory environment - that capability
simply does not yet exist. \\
Speaking to shared memory parallization, many functions which are part
of the ADER suite are used in a threaded manner but are not labeled as
thread-safe. This is because much of ADER's parallization happens on a
per-cluster basis, each thread is given one cluster of materials to work with;
that work being building or solving a matrix. Many of these threaded functions
would fail to work as intended if they were not parallelized by the material
cluster which they are operating on. Only those functions which are fully
thread-safe, generally those which do not alter program memory, are labeled as
such. There are two branching points for ADER threads. In
\texttt{ADERCorrectTransportCycle} each material cluster is handed off to a
thread which will then fill and solve the cluster composition optimization
problem. Following the solution of these composition matrices a
OpenMP barrier prevents any thread from moving forward until
all threads have solved all their optimization problems afterwhich program flow
is reduced back down to a single thread. In 
\texttt{ADERProcessMaterialAderData} each material is handed off to a thread
which then proceeds to fill in that material's composition matrix information.
Again, an OpenMP barrier prevents program continuation past this point until
control is returned back to a single thread. 

\end{document}
